\section{Сбалансированные, 2-3 и красно-черные деревья}

\subsection*{Сбалансированные деревья}
Идеально сбалансированное дерево = все пути от корня до
конечных вершин имеют одинаковую длину.
В реальности так не бывает почти никогда(!), поэтому хочется, чтобы min/max длины отличались не сильно.

Два способа этого достичь:

\begin{itemize}
\item AVL (Классическое определение сбалансированного дерева)
Для любого узла разница между высотами левого и правого
поддерева не превосходит 1.
\item Красно-черные Количество черных ребер в любом пути одинаково,
количество красных ребер в пути не превосходит количество
черных + 1.
\end{itemize}

Отсюда гарантированная сложность - $O(logN)$

\subsection*{2-3 деревья}

{\bf Структура:} каждый узел может содержать 1 ключ и $\le 2$ потомков или 2 ключа и $\le 3$ потомков. Ключи отсортированны внутри узла. В 2-узле левое поддерево содержит ключи меньше ключа в узле, правое - больше. В 3-узле левое поддерево содержит ключи меньшие левого ключа узла, правое - большие правого ключа, среднее - большие левого, но меньшие правого.

{\bf Вставка:} 

1. Выполняем поиск, находим лист, в который надо
вставить ключ.

2. Если лист является 2-узлом, то превращаем его в 3-узел.

3. Если лист является 3-узлом, то превращаем в 4-узел.

4. 4-узел хранит 3 ключа – средний ключ перемещаем в
родителя, левый и правый ключи превращаем в 2-узлы.

5. Если родитель был 3-узлом, то он превратился в 4-узел
– выполняем для него шаг 4.
 
{\bf Удаление:} (Не помню, чтобы было на лекциях, однако алгоритм такой же как в B-дереве в билете 9, так как 2-3 дерево - частный случай B-дерева с M = 3).

\subsection*{Красно-черные деревья}
Идея: в 2-3 дереве заменить 3-узлы на 2 узлы с дополнительным "красным" ребром (идущим, например, влево для левостороннего к-ч дерева).

{\bf Свойства:}
\begin{itemize}
\item Количество черных ребер в любом пути одинаково
\item Все красные ребра идут налево
\item Нет двух последовательных красных ребер
\item Взаимнооднозначно соответствуют 2-3 деревьям
\item Реализация get – как в обычном BST
\end{itemize}

{\bf Вставка:}

Алгоритм будет показан как программная реализация:

\begin{verbatim}
private static final boolean RED = true;
private static final boolean BLACK = false;

private class Node
{
    Key key;
    Value val;
    Node left, right;
    boolean color; // color of parent link
}

private boolean isRed(Node x)
{
    if (x == null) return false;
    return x.color == RED;
}
\end{verbatim}

3 вспомогательных операции:


\begin{verbatim}
private Node rotateLeft(Node h)
{
    // assert isRed(h.right);
    Node x = h.right;
    h.right = x.left;
    x.left = h;
    x.color = h.color;
    h.color = RED;
    return x;
}
\end{verbatim}


\begin{verbatim}
private Node rotateRight(Node h)
{
    // assert isRed(h.left);
    Node x = h.left;
    h.left = x.right;
    x.right = h;
    x.color = h.color;
    h.color = RED;
    return x;
}

\end{verbatim}


\begin{verbatim}
private void flipColors(Node h)
{
    // assert !isRed(h);
    // assert isRed(h.left);
    // assert isRed(h.right);
    h.color = RED;
    h.left.color = BLACK;
    h.right.color = BLACK;
}
\end{verbatim}

Задача - сохранить свойства к-ч дерева при вставке, из этого вытекает следующая функция вставки:

\begin{verbatim}
private Node put(Node h, Key key, Value val)
{
    if (h == null) return new Node(key, val, RED);
    
    int cmp = key.compareTo(h.key);
    if (cmp < 0) h.left = put(h.left, key, val);
    else if (cmp > 0) h.right = put(h.right, key, val);
    else if (cmp == 0) h.val = val;
    
    // Все красные ребра идут налево
    if (isRed(h.right) && !isRed(h.left)) h = rotateLeft(h);
    // Не может быть два красных ребра подряд     
    if (isRed(h.left) && isRed(h.left.left)) h = rotateRight(h);
    // Все красные ребра идут налево 
    if (isRed(h.left) && isRed(h.right)) flipColors(h);          
    
    return h;
}
\end{verbatim}

{\bf Удаление:} (На лекциях не рассказывалось)

