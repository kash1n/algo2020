\section{Распределенные хеш-таблицы. Фильтр Блума.}
\noindent{\bf Распределенной хеш-таблицей} называется такая таблица, которая распределена на разные узлы (сервера).
Имеется проблема: когда отказывает узел, то в обычной таблице перестраивается все. Как устранить эту проблему в распределенной таблице?
\subsection {Консистентное (согласованное) хеширование}
Берем наши обьекты и распределяем их по единичному кругу (согласно хеш-функции). Вклиниваем равномерно наши сервера между этими обьектами. Идем по часой стрелки от перого обьекта. Первый встречный узел - владелец этого обьекта. В таком случае, при отключении или добавлении нового узла перестраивается лишь небольшая часть таблицы.\\
\begin{tabular}{|c|c|c|}
\hline
\multicolumn{3}{|c|}{m - размер таблицы, N - число узлов}       \\ \hline
              & Обычная хеш-таблица & Согласованное хеширование \\ \hline
Добавить узел & O(m)                & O(m/N + logN)             \\ \hline
Удалить узел  & O(m)                & O(m/N + logN)             \\ \hline
Добавить ключ & O(1)                & O(logN)                   \\ \hline
Удалить ключ  & O(1)                & O(logN)                   \\ \hline
\end{tabular}

\subsection {Рандеву хеширование}
Рассматриваем следующую хеш-функцию: $h: U \times S \rightarrow {0, ..., m}$. S - множество всех узлов. Изначально разбрасываем обьекты на узлы с максимальным весом, т.е. 
$u \rightarrow S_{dst}$, где $h(u, S_{dst}) = \min h(u, S_i)$. Очевидно, если узел был удален, или добавлен, то почти ничего не меняется.


\subsection {Фильтр Блума}
Пусть имеется $m$ бит (фильтр, mask) и $k$ хеш-функций. Считаем, что хеш-функции - независимые случайные величины, такие, что 
$$P(h_s(u) = i) = \frac{1}{m}$$

\begin{itemize}
\item Вставка: Записываем $mask[h_s(u)] = 1$, $\forall s$.
\item Поиск по ключу: Если $mask[h_s(u)] == 1$, $\forall s$, то элемент присутствует в хеш-таблице.
\end{itemize}

Вероятность того, что в $i$-ый бит не будет записана единица во время вставки элемента есть 
$$P(h_1(u) \ne i)...P(h_s(u) \ne i)...P(h_k(u) \ne i) = \left( 1 - \frac{1}{m}\right)^k$$

Вероятность того, что $i$-ый бит останется равным нулю после вставки $n$ различных элементов в изначально пустой фильтр Блума есть 
$$\left( 1 - \frac{1}{m}\right)^{nk} \approx e^{-\frac{kn}{m}}$$

Ложноположительное срабатывание состоит в том, что для некоторого элемента $u$, не равного ни одному из вставленных, все $k$ бит с номерами $h_s(u)$ окажутся ненулевыми, и фильтр Блума ошибочно ответит, что $u$ входит во множество вставленных элементов. Вероятность такого события примерно равна
$$(1 - e^{-\frac{kn}{m}})^k$$

Эта вероятность падает с ростом $m$ (размер фильтра) и растет с ростом $n$ (число вставленных элементов).
