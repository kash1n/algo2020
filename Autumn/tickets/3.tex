\section {Алгоритм Карацубы умножения чисел, алгоритм Штрассена умножения матриц. Быстрое преобразование Фурье}

\noindent{\bf Алгоритм Карацубы} \\
Пусть есть два n-значных двоичных числа $A = a x + b$ и $B=c x + d$, где n — чётное число и $x=10^{\frac n 2}$. То есть, a и c получены из старших $\frac n 2$ разрядов A и B, а b и d — из младших. \\
\noindent В таких обозначениях произведение чисел A и B может быть переписано как
$$
AB = (a x + b)(c x + d) = a c x^2 + (a d + b c) x + b d
$$

\noindent Таким образом, умножение n-значных чисел было сведено к четырём задачам умножения $\frac n 2$-значных чисел и сложению результатов, которые выполняются за $O(n)$.
Далее можем заметить, что на самом деле достаточно лишь трёх умножений $\frac n 2$-значных чисел, так как
$$
(a d + b c) = (a + b)(c + d) - a c - b d
$$
Таким образом, всё произведение AB может быть получено из $a c$, $b d$ и $(a + b)(c + d)$ линейными операциями сложения, вычитания, а общее время работы оценивается как
$$
T(n) = 3 T (\frac {n}{2}) + O(n) =  O(n^{\log _{2}3})
$$

\noindent{\bf Алгоритм Штрассена} 

\noindent Хотим перемножить две матрицы. Основная идея: сводим умножение двух матриц размера $n$ к семи умножениям матриц размерности $\frac n 2$
\begin{equation*}
XY = \left(
\begin{array}{cc}
A & B \\
C & D\\
\end{array}
\right)
\left(
\begin{array}{cc}
E & F \\
G & H \\
\end{array}
\right)
=
\left(
\begin{array}{cc}
AE + BG & AF + BH \\
CE + DG & CF + DH \\
\end{array}
\right)
\end{equation*}

\noindent Можно заметить, что 
\begin{equation*}
XY = \left(
\begin{array}{cc}
P_5 + P_4 - P_2 + P_6 & P_1 + P_2 \\
P_3 + P_4 & P_1+ P_5 - P_3 - P_7\\
\end{array}
\right)
\end{equation*}

\begin{equation*}
\begin{array}{llllll}
P_1 &=& A(F - H) & P_2 &=& (A + B)H \\
P_3 &=& (C + D)E & P_4 &=& D(G - E) \\
P_5 &=& (A + D)(E + H) & P_6 &=& (B - D)(G + H) \\
P_7 &=& (A - C)(E + F) &  \\
\end{array}
\end{equation*}

\noindent В терминах основной теоремы из второго билета имеем (т.к. сложение матриц - это $n^2$):
$$
T(n) = 7T(\frac{n}{2}) + O(n^2)
$$
\noindent Поэтому сложность равна $O(n^{\log_2 7})$

\noindent{\bf Быстрое преобразование Фурье} \\
Теперь хотим перемножать многочлены. Пусть даны $A(x) = a_0 + a_1 x + .. + a_n x^n$ и $B(x) = b_0 + b_1 x + .. + b_n x^n$. Их произведение обозначим за $C(x).$ \\

\noindent Заметим, что если многочлены заданы не в виде набора коэффициентов, а в виде значений в (2n + 1) различных точках, то умножение многочленов работает за линейное время (2n + 1, так как многочлен-произведение имеет степень 2n). Действительно, пусть $A(p_i) = \alpha_i$, $B(p_i) = \beta_i$, тогда $C(p_i) = A(p_i) B(p_i) = \alpha_i * \beta_i$ \\

\noindent Теперь если мы научимся переходить от коэффициентов многочлена к значениям и, наоборот, по значениям вычислять коэффициенты за $O(n \log n)$, то итоговое умножение тоже будет работать за $O(n \log n)$ \\

\noindent Заметим, что
\begin{equation*}
\begin{array}{lll}
A(x) &=& A_0(x^2) + x A_1(x^2)\text{, где} \\
A_0(X) &=& a_0 + a_2 X + a_4 X^2 + ... \\
A_1(X) &=& a_1 + a_3 X + a_5 X^2 + ...
\end{array}
\end{equation*} \\

\noindent Так как в представлении A мы считаем $A_0$ и $A_1$ от $x^2$, то выбрав симметричные относительно нуля точки мы ускоримся в подсчете значений по коэффициентам в два раза (т.к при подсчете $A(x)$ и $A(-x)$ вычисляются одни и те же $A_0(x^2)$ и $A_1(x^2)$) \\

\noindent Но для применения основной теоремы из второго билета это сработает только один раз, и перейти от $\frac n 2$ к $\frac n 4$ уже не получится, поскольку все точки теперь положительные. Проблема решается выходом в комплексную плоскость и вычислением значений многочлена A(x) в корнях из единицы степени n (там каждый раз половина точек будет склеиваться и к концу как раз останется одна точка - единица) \\

\noindent Теперь рассмотрим переход от значений в корнях из единицы к коэффициентам многочлена (обратное преобразование Фурье). Давайте перепишем то, что мы сделали (переход от коэффициентов к значениям, оно же прямое преобразование Фурье) в матричном виде:

\begin{equation*}
\left(
\begin{array}{c}
A(1) \\
A(\omega) \\
A(\omega^2) \\
\vdots \\
A(\omega^n) 
\end{array}
\right)
= 
\left(
\begin{array}{cccc}
1 & 1 & \ldots & 1 \\
1 & \omega & \ldots & \omega^n \\
1 & \omega^2 & \ldots & \omega^{2n}\\
\vdots & \vdots & \ddots & \vdots\\
1 & \omega^{n} & \ldots & \omega^{n^2} \\
\end{array}
\right)
\left(
\begin{array}{c}
a_0 \\
a_1 \\
a_2 \\
\vdots \\
a_n 
\end{array}
\right)
= M_n (\omega)
\left(
\begin{array}{c}
a_0 \\
a_1 \\
a_2 \\
\vdots \\
a_n 
\end{array}
\right)
\end{equation*}
Чтобы перейти от значений к коэффициентам нужно умножить слева это на $M_n^{-1} (\omega)$. Непосредственным умножением проверяется, что $M_n^{-1} (\omega) = \frac{1}{n} M_n (-\omega)$. То есть в матричном виде принципиально ничего не поменялось (появилось умножение на $\frac{1}{n}$ и $-w$ вместо $w$, но это никак принципиально не влияет на тот алгоритм, который был показан). Поэтому этот же алгоритм применим и для получения коэффициентов по значениям.

\noindentВ терминах основной теоремы имеем (для преобразований Фурье):
$$
T(n) = 2T(\frac n 2) + O(n)
$$
То есть общая сложность $O(n \log n)$ \\

\noindent {\bf P.S.} Это все работает, когда n - степень двойки. Если это не так, то просто дополним старшие коэффициенты многочлена нулями до ближайшей степени двойки. Очевидно, что степень многочлена увеличится не более чем в два раза, что не влияет на ассимптотику, т.к $O(2 n \log(2 n)) = O(n \log n)$
