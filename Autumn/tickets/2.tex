
\section {Сортировка слиянием. Основная теорема для метода «разделяй и властвуй»}
\noindent{\bf Сортировка слиянием (Merge sort):}\\
В основе лежит идея «Разделяй и властвуй» (Divide and conquer):
\begin{enumerate}
\item Разделим массив на 2 части размера $\frac{n}{2}$
\item Отсортируем обе части (2 рекурсивных вызова)
\item Выполним процедуру слияния: объединяем отсортированные части таким образом, чтобы
получить полностью отсортированный массив
\end{enumerate}
Ключевая процедура - слияние (merge). Особенность: слияние требует дополнительный массив
\texttt{aux} (auxiliary - вспомогательный). Выделяем его сразу, чтобы не делать этого при
каждом вызове.
Код:
\begin{verbatim}
procedure MergeSort( a[] ):
    aux = new array [0..n-1]
    MergeSort(a, aux, 0, n-1)
end

procedure MergeSort( a[], aux[], low, high ):
    if (high <= low):
        return
    mid = low + (high - low) / 2
    MergeSort(a, aux, low, mid)
    MergeSort(a, aux, mid + 1, high)
    Merge(a, aux, low, mid, high)
end

procedure Merge( a[], aux[], low, mid, high ):
    for k in [low..high]:
        aux[k] = a[k]
    i = low, j = mid + 1
    for k in [low..high]:
        if i > mid:
            a[k] = aux[j++]
        else if j > high:
            a[k] = aux[i++]
        else if aux[j] < aux[i]:
            a[k] = aux[j++]
        else:
            a[k] = aux[i++]
    end
end
\end{verbatim}
Анализ сложности будет проведен позже.\\
{\bf Основная теорема (Master theorem):}\\
Пусть $n$ - размер задачи. $a$ - количество подзадач в рекурсии. $\frac{n}{b}$ - размер каждой подзадачи. $O(n^d)$ - оценка сложности работы, производимой алгоритмом вне рекурсивных вызовов. Пусть
$$
T(n) = aT(\left\lceil\frac{n}{b}\right\rceil) + O(n^d),\ \ \ a>0,\ b>1,\ d\ge0
$$
Тогда
$$
T(n) = 
\begin{cases}
 O(n^d),\ d>\log_{b}a \\
 O(n^d\log n),\ d=\log_{b}a \\
 O(n^{\log_{b}a}),\ d<\log_{b}a \\
\end{cases}
$$
Пользуясь этой теоремой, оценим сложность MergeSort. 
$$
T(n) = 2T(\frac{n}{2}) + O(n)
$$ 
В данном случае $a = 2,\ b = 2,\ d = 1$. Поэтому мы имеем дело со вторым случаем Основной теоремы, и $T(n) = O(n\log n)$.
