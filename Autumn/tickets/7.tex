\section {Ассоциативные массивы, бинарные деревья поиска.}
\noindent{\bf Ассоциативный массив (Map, Dictionary)}\\
Абстрактный тип данных:

\qquad набор пар <ключ, значение> (ключ уникальный).

Операции (интерфейс):
\begin{enumerate}
\item Вставка
\item Поиск по ключу
\item Удаление по ключу
\end{enumerate}

\begin{verbatim}
interface Map<K, V> {
    ...
    V get(K key);
    V put(K key, V value);
    V remove(K key);
    ...
}
\end{verbatim}

{\bf Реализация ассоциативного массива}
\begin{enumerate}
\item Массив
\item Отсортированный массив
\item Бинарные деревья поиска
\item Хеш-таблицы
\end{enumerate}
{\bf C\#: интерфейс IDictionary}
\begin{verbatim}
public interface IDictionary<TKey, TValue> :
    ICollection<KeyValuePair<TKey, TValue>>,
    IEnumerable<KeyValuePair<TKey, TValue>>
{
    ICollection<TKey> Keys { get; }
    ICollection<TValue> Values { get; }
    
    TValue this[TKey key] { get; set; }
    void Add(TKey key, TValue value);
    bool ContainsKey(TKey key);
    bool Remove(TKey key);
    bool TryGetValue(TKey key, out TValue value);
}

public interface ICollection<T> : IEnumerable<T>, IEnumerable
{
    int Count { get; }
    void Clear();
    ...
}
\end{verbatim}
{\bf Java: интерфейс Map}
\begin{verbatim}
public interface Map<K,V> {
    int size();
    boolean isEmpty();
    boolean containsKey(Object key);
    boolean containsValue(Object value);
    V get(Object key);
    V put(K key, V value);
    V remove(Object key);
    void clear();

    Set<K> keySet();
    Collection<V> values();
    Set<Map.Entry<K, V>> entrySet();
    
    interface Entry<K,V> {
        K getKey();
        V getValue();
        V setValue(V value);
    }
}
\end{verbatim}
{\bf Java: интерфейс Set}
\begin{verbatim}
public interface Set<E> implements Iterable<E> {
    int size();
    boolean isEmpty();
    boolean contains(Object o);
    V get(Object key);
    boolean add(E e);
    V remove(Object o);
    void clear();
    ...
}
\end{verbatim}
{\bf Стандартные классы в Java и C\#}\\
\underline{Java:} \\
HashSet<T>, HashMap<K, V> \\
TreeSet<T>, TreeMap<K, V>

\begin{verbatim}
public abstract class AbstractSet<E> implements Set<E>
public class HashSet<E> extends AbstractSet<E>
public class TreeSet<E> extends AbstractSet<E>

public abstract class AbstractMap<K,V> implements Map<K,V>
public class HashMap<K,V> extends AbstractMap<K,V>
public class TreeMap<K,V> extends AbstractMap<K,V>
\end{verbatim}
\underline{C\#:} \\
HashSet<T>, Dictionary<K,V> \\
SortedSet<T>, SortedDictionary<K,V>\\
{\bf Подсчет количества слов}
\begin{verbatim}
private static void CountWords()
{
    var dic = new Dictionary<string, int>();
    while (true)
    {
        string line = Console.ReadLine();
        if (string.IsNullOrEmpty(line))
        break;
        
        string[] words = line.Split(’ ’);
        foreach (string word in words)
        {
            if (dic.ContainsKey(word))
                dic[word] = dic[word] + 1;
            else
                dic[word] = 1;
        }
    }
    
    foreach (var kvp in dic)
        Console.WriteLine("{0} = {1}", kvp.Key, kvp.Value);
}
\end{verbatim}

\noindent{\bf Бинарное дерево поиска}\\
\textit{Двоичное дерево поиска} (англ. binary search tree, BST) — это двоичное дерево, для которого выполняются следующие дополнительные условия (свойства дерева поиска):

\begin{enumerate}
\item Оба поддерева — левое и правое — являются двоичными деревьями поиска.
\item У всех узлов \textit{левого} поддерева произвольного узла X значения ключей данных \textit{меньше либо равны}, нежели значение ключа данных самого узла X.
\item У всех узлов \textit{правого} поддерева произвольного узла X значения ключей данных \textit{больше либо равны}, нежели значение ключа данных самого узла X.
\end{enumerate}
(Ключи уникальны, поэтому отношение строго <, наверное)\\
Вершины - записи вида (data, left, right), иногда еще нужен parent, data - пара (key, value)\\
Базовый интерфейс двоичного дерева поиска состоит из трёх операций (сложность из википедии):
\begin{enumerate}
\item FIND(K) — поиск узла, в котором хранится пара (key, value) с key = K. (Сложность в среднем  O(log n), в худшем случае~-- O(n))
\item INSERT(K, V) — добавление в дерево пары (key, value) = (K, V). (В ср.~-- O(log n), в худ~-- O(n))
\item REMOVE(K) — удаление узла, в котором хранится пара (key, value) с key = K. (В ср.~-- O(log n), в худ~-- O(n))
\end{enumerate}
{\bf Поиск элемента (FIND)}\\
\textit{Дано:} дерево Т и ключ K.\\
\textit{Задача:} проверить, есть ли узел с ключом K в дереве Т, и если да, то вернуть ссылку на этот узел.\\
\textit{Алгоритм:}
\begin{itemize}
\item Если дерево пусто, сообщить, что узел не найден, и остановиться.
\item Иначе сравнить K со значением ключа корневого узла X.
\begin{itemize}
\item[\labelitemi] Если K=X, выдать ссылку на этот узел и остановиться.
\item[\labelitemi] Если K>X, рекурсивно искать ключ K в правом поддереве Т.
\item[\labelitemi] Если K<X, рекурсивно искать ключ K в левом поддереве Т.
\end{itemize}
\end{itemize}
{ \bf Добавление элемента (INSERT)}\\
\textit{Дано:} дерево Т и пара (K, V).\\
\textit{Задача:} вставить пару (K, V) в дерево Т (при совпадении K, заменить V).\\
\textit{Алгоритм:}
\begin{itemize}
\item Если дерево пусто, заменить его на дерево с одним корневым узлом ((K, V), null, null) и остановиться.
\item Иначе сравнить K с ключом корневого узла X.
\begin{itemize}
\item[\labelitemi] Если K>X, рекурсивно добавить (K, V) в правое поддерево Т.
\item[\labelitemi] Если K<X, рекурсивно добавить (K, V) в левое поддерево Т.
\item[\labelitemi] Если K=X, заменить V текущего узла новым значением.
\end{itemize}
\end{itemize}
{ \bf Удаление узла (REMOVE)}\\
\textit{Дано:} дерево Т с корнем n и ключом K.\\
\textit{Задача:} удалить из дерева Т узел с ключом K (если такой есть).\\
\textit{Алгоритм:}\\
 \noindent 1. Если дерево T пусто, остановиться;\\
2. Иначе сравнить K с ключом X корневого узла n.
\begin{itemize}
\item[\labelitemi] Если K>X, рекурсивно удалить K из правого поддерева Т;
\item[\labelitemi] Если K<X, рекурсивно удалить K из левого поддерева Т;
\item[\labelitemi] Если K=X, то необходимо рассмотреть три случая.
\begin{itemize}
\item[\labelitemi] Если обоих детей нет, то удаляем текущий узел и обнуляем ссылку на него у родительского узла;
\item[\labelitemi] Если одного из детей нет, то значения полей ребёнка m ставим вместо соответствующих значений корневого узла, затирая его старые значения, и освобождаем память, занимаемую узлом m;
\item[\labelitemi] Если оба ребёнка присутствуют, то
\begin{itemize}
\item[\labelitemi] Если самый левый элемент правого поддерева m не имеет поддеревьев
\begin{itemize}
\item[\labelitemi] Копируем значения K, V из m в удаляемый элемент
\item[\labelitemi] Удаляем m
\end{itemize}
\item[\labelitemi] Если m имеет правое поддерево
\begin{itemize}
\item[\labelitemi] Копируем значения K, V из m в удаляемый элемент
\item[\labelitemi] Заменяем у родительского узла ссылку на m ссылкой на правое поддерево m
\item[\labelitemi] Удаляем m
\end{itemize}
\end{itemize}
\end{itemize}
\end{itemize}
{\bf Реализация ("Algorithms in Java")}
\begin{verbatim}
public class BST<Key extends Comparable<Key>, Value>
{
    private class Node
    {
        private Key key; // key
        private Value val; // associated value
        private Node left, right; // links to subtrees
        private int N; // # nodes in subtree rooted here
        
        public Node(Key key, Value val, int N) {
            this.key = key; this.val = val; this.N = N;
        }
    }
    
    private Node root; // root of BST
    
    public int size() { return size(root); }
    private int size(Node x) { return x == null ? 0 : x.N; }
    
    public Value get(Key key) { ... }
    public void put(Key key, Value val) { ... }
}
\end{verbatim}
Бинарные деревья поиска используют сравнения
\begin{verbatim}
interface Comparable<T> {
int compareTo(T o);
}
public interface Comparator<T> {
int compare(T o1, T o2);
}
\end{verbatim}
\noindent Java: Comparable, Comparator \\
C\#: IComparable, IComparer\\
{\bf Реализация: get}
\begin{verbatim}
public Value get(Key key) {
    return get(root, key);
}

private Value get(Node x, Key key) {
    // Return value associated with key in the subtree rooted at x;
    // return null if key not present in subtree rooted at x.
    if (x == null) return null;
    
    int cmp = key.compareTo(x.key);
    if (cmp < 0) return get(x.left, key);
    else if (cmp > 0) return get(x.right, key);
    else return x.val;
}
\end{verbatim}
{\bf Реализация: put}
\begin{verbatim}
public void put(Key key, Value val) {
    // Search for key. Update value if found; grow table if new.
    root = put(root, key, val);
}

private Node put(Node x, Key key, Value val) {
    // Change key’s value to val if key in subtree rooted at x.
    // Otherwise, add new node to subtree associating key with val.
    if (x == null) return new Node(key, val, 1);
    
    int cmp = key.compareTo(x.key);
    if (cmp < 0) x.left = put(x.left, key, val);
    else if (cmp > 0) x.right = put(x.right, key, val);
    else x.val = val;
    
    x.N = size(x.left) + size(x.right) + 1;
    return x;
}
\end{verbatim}
(Тут какая-то дичь с симуляциями, думаю, что это лучше прямо в презенташке смотреть)\\
{\bf Сложность в случайном BST}\\
\textit{Утв.} Search hit в случайном бинарном дереве поиска,
содержащем N ключей требует в среднем $\sim 2 ln N (\approx 1.39 log N)$ сравнений.\\
%\vspace{3mm}
Пусть $C_N$ – сумма длин путей до всех узлов в дереве с $N$ узлами (тогда успешный поиск требует в среднем $1 + C_N /N$).
\begin{center}
$C_N = N-1+(C_0+C_{N-1})/N+(C_1+C_{N-2})/N+. . .+(C_{N-1}+C_0)/N$\\
$C_0 = C_1 = 0$
\end{center}
(рекурентное соотношение аналогично тому, которое возникало при анализе Quicksort, см. лекцию 3).
\vspace{2mm}
\begin{center}
$C_N \sim 2N ln N$
\end{center}
\vspace{3mm}
\textit{Утв.} Вставка и search miss в случайном бинарном дереве
поиска, содержащем $N$ ключей требует в среднем $\sim 2 ln N (\approx 1.39 log N)$ сравнений.\\
Вставка и search miss требует на 1 сравнение больше, чем
search hit (задача).

\vspace{3mm}
\noindent{\bf Hibbard deletion (Удаление элемента)}\\
Один из эффективных способов сделать это, обнаруженный Томасом Хиббардом в 1962 году, состоит в том, чтобы заменить удаляемый узел местами его преемника . Преемником ключа является следующий по величине ключ в дереве, и его удобно найти как минимальный ключ в правом поддереве удаляемого узла.

Так же в презенташке написано Tombstone в удалении, но я толком не нашла ничего.
