\section{Очередь с приоритетами, реализация с помощью бинарной кучи. Пирамидальная сортировка. Алгоритм Дейкстры.}
\noindent{\bf Очередь с приоритетами:}\\
Очередь с приоритетом (priority queue) - абстрактный тип данных, поддерживающий две обязательные операции - добавить элемент и извлечь максимум (минимум). Предполагается, что для каждого элемента можно вычислить его {\it приоритет} - действительное число или в общем случае элемент линейно упорядоченного множества.\\
Операции (интерфейс):
\begin{itemize}
\item Добавить элемент (insert)
\item Исключить минимальный элемент (deleteMin)
\end{itemize}
Дополнительные операции (интерфейс):
\begin{itemize}
\item Построить очередь (makeQueue / buildHeap)
\item Уменьшить значение ключа (decreaseKey)
\end{itemize}
Варианты реализации:
\begin{enumerate}
\item Неупорядоченный массив.\\
insert: $O(1)$ - просто добавляем в конец\\
deleteMin: $O(n)$ - надо перебрать все элементы
\item Упорядоченный массив (по убыванию).\\
insert: $O(n)$ - приходится сдвигать кусок массива\\
deleteMin: $O(1)$ - просто берем последний элемент
\item {\bf Бинарная куча (Heap, Binary heap).}\\
insert: $O(\log n)$\\
deleteMin: $O(\log n)$\\
buildHeap: $O(n)$\\
decreaseKey: $O(\log n)$
\end{enumerate}
Если кратко, бинарная куча - массив следующего вида:\\
для элемента с индексом $i$: индексы "детей"{} = \{$2i$, $2i + 1$\}, индекс "родителя"{} = $i/2$.\\
Бинарную кучу следует представлять себе как дерево, для которого выполнены три условия:
\begin{itemize}
\item Значение в любой вершине не меньше, чем значения её потомков.
\item Глубина всех листьев (расстояние до корня) отличается не более чем на 1 слой.
\item Последний слой заполняется слева направо без «дырок».
\end{itemize}
У бинарной кучи существуют операции swim (всплытие узла) и sink (погружение узла).\\
Код:
\begin{verbatim}
h = array of keys of length N

procedure swim(i):
    while i > 1 and h[i/2] < h[i]:
        h[i/2], h[i] = h[i], h[i/2]
        i = i / 2
    end
end

procedure sink(i):
    while 2*i <= N:
        int k = argmin(h[2*i], h[2*i + 1])
        if h[k] < h[i]:
            h[k], h[i] = h[i], h[k]
        else:
            break
    end
end
\end{verbatim}
Построение кучи, которое описано ниже, соответствует операции makeQueue и выполняется за $O(n)$:
\begin{verbatim}
N = array length
h = array of keys of length N

procedure buildHeap():
    for i = N to 1:
        sink(i)
end
\end{verbatim}
Подробнее и более понятно о бинарной куче: \href{https://clck.ru/ScTWD}{Двоичная куча}\\

\noindent{\bf Heapsort (пирамидальная сортировка):}\\
Процедура Heapsort сортирует массив (бинарную кучу) без привлечения дополнительной памяти за
$O(n \log n)$.\\
Будем удалять элементы из корня по одному за раз и перестраивать дерево. То есть на первом шаге обмениваем \texttt{h[0]} и \texttt{h[N-1]}, преобразовываем \texttt{h[0], h[1], ... , h[N-2]} в бинарную кучу. Затем переставляем \texttt{h[0]} и \texttt{h[N-2]}, преобразовываем \texttt{h[0], h[1], ... , h[N-3]} в бинарную кучу и т.д. Процесс продолжается до тех пор, пока в куче не останется один элемент. Тогда \texttt{h[0], h[1], ... , h[N-1]} - упорядоченная последовательность.\\
\note{Нужно, наверное, написать подробнее про Heapsort, но на лекциях этого вообще толком не было...}\\
{\bf Алгоритм Дейкстры:}\\
Пусть на ребрах графа заданы длины: $G = (V, E),\ l : E \rightarrow \mathbf{R}_{+}$\\
Задача: найти кратчайшие расстояния до всех вершин из данной вершины $s$.\\
Код:
\begin{verbatim}
procedure Dijkstra( G, s ):
    dist = array of size G.V
    prev = array of size G.V

    for x in G.V:
        dist[x] = inf
        prev[x] = -1

    dist[s] = 0
    Q = priority queue
    Q.makeQueue(V, dist)

    while not Q.isEmpty:
        x = Q.deleteMin()
        for y in G.neighbours(x):
            key = dist[x] + l(x, y)
            if dist[y] > key:
                dist[y] = key
                prev[y] = x
                Q.decreaseKey(y, key)
            end
        end
    end
end
\end{verbatim}
Подробнее: \href{https://ru.wikipedia.org/wiki/Алгоритм_Дейкстры}{Википедия}
