%main.tex
\documentclass[specialist, subf, href, colorlinks=true, 12pt, times, mtpro, final]{disser}
\usepackage[utf8]{inputenc}
\usepackage[english,russian]{babel}
\usepackage{a4wide}
\usepackage{mathtext}
\usepackage{amsbsy}
\usepackage{amsthm}
\usepackage{amsmath}
\usepackage{amssymb}
\usepackage{amsfonts}
\usepackage{tikz}
\usepackage{verbatim}
\usepackage{graphicx}

\definecolor{faded}{gray}{0.6}
\def\note{\textcolor{faded}}
 
\begin{document}

\tableofcontents

\newpage
{\Large \bf Программа экзамена (осень 2020)}

\begin{enumerate}
{\footnotesize
\item Сложность алгоритмов, бинарный поиск, сортировка выбором и вставками
\item Сортировка слиянием. Основная теорема для метода «разделяй и властвуй».
\item Алгоритм Карацубы умножения чисел, алгоритм Штрассена умножения матриц. Быстрое преобразование Фурье.
\item Оценка снизу количества сравнений при сортировке, быстрая сортировка Хоара, сложность в среднем и в худшем. Задача Дейкстры о голландском флаге, 3х-частное разбиение, эвристики выбора опорного элемента.
\item Алгоритмы нахождения k-й порядковой статистики - вероятностный и детерминированный. Метод сортировки TimSort.
\item Абстрактные типы данных, интерфейс и реализация, стек и очередь - реализация связанным списком и массивом.
\item Ассоциативные массивы, бинарные деревья поиска.
\item Сбалансированные деревья, 2-3 и красно-черные деревья.
\item B-деревья.
\item Хеш-таблицы, реализация методом цепочек и открытой адресацией.
\item Распределенные хеш-таблицы. Фильтр Блума.
\item Графы, способы представления в программе. Поиск в глубину и поиск в ширину.
\item Топологическая сортировка. Алгоритм Косарайю поиска сильносвязанных компонент.
\item Очередь с приоритетами, реализация с помощью бинарной кучи. Пирамидальная сортировка. Алгоритм Дейкстры.
}
\end{enumerate}

\section {Сложность алгоритмов, бинарный поиск, сортировка выбором и вставками}
\noindentПусть $\Omega_{n}$ – множество всех допустимых наборов входных данных размера $n$.\\
$T(\omega)$ – {\bf сложность} (количество операций) алгоритма $A$ на входных данных $\omega \in \Omega_{n}$.\\
Сложность {\bf в худшем случае}:
$$
  T(n) = \max\limits_{\omega \in \Omega_{n}}T(\omega)
$$
Сложность {\bf в среднем (average)}:
$$
  T_{avg}(n) = \frac{1}{|\Omega_{n}|}\sum\limits_{\omega \in \Omega_{n}}T(\omega)
$$
\note{// Отступление про $O$-нотацию.}\\
{\bf Поиск элемента в массиве (бинарный поиск):}\\
Дано: отсортированный массив \texttt{a[]}, элемент \texttt{x}\\
Требуется: определить, содержится ли \texttt{x} в \texttt{a[]}\\
Идея:
\begin{enumerate}
\item Делим массив пополам
\item Сравниваем \texttt{x} с центральным элементом \texttt{a[mid]}
\begin{itemize}
\item \texttt{x = a[mid]}? – нашли
\item \texttt{x < a[mid]}? – ищем в нижней части массива
\item \texttt{x > a[mid]}? – ищем в верхней части массива
\end{itemize}
\end{enumerate}
Код:\\
\begin{verbatim}
procedure BinarySearch( a[], x ):
    // a[] - array from 0 to n-1
    low = 0
    high = n-1
    while low <= high:
        mid = low + (high - low) / 2
        if a[mid] < x:
            low = mid + 1
        else if a[mid] > x:
            high = mid - 1
        else
            return mid
    end
    return -1
end
\end{verbatim}
Пусть $n \le 2^k$ ($k = [\log n]$). Итераций: $k + 1$. На каждой итерации: $O(1)$. Итого:
$O(\log n)$.\\
{\bf Сортировка выбором (Selection sort):}\\
Шаги алгоритма:
\begin{enumerate}
\item Находим номер минимального (или максимального, смотря что хотим) значения в текущем списке
\item Производим обмен этого значения со значением первой неотсортированной позиции (обмен не нужен, если минимальный элемент уже находится на данной позиции)
\item Теперь сортируем хвост списка, исключив из рассмотрения уже отсортированные элемент
\end{enumerate}
Код:
\begin{verbatim}
procedure SelectionSort( a[] ):
    // a[] - array from 0 to n-1
    for i in [1..n-1]:
        j = index of max element in a[0..n-i]
    swap (a[j], a[n-i])
    end
end
\end{verbatim}
Количество сравнений:
$$
(n-1)+(n-2)+...+1 = \frac{n(n-1)}{2}
$$
Итого: $O(n^2)$.\\
{\bf Сортировка вставками (Insertion sort):}\\
Идея: для каждого элемента ищем место в отсортированной части массива, в которое его нужно вставить.\\
Код:
\begin{verbatim}
procedure InsertionSort( a[] ):
    for i in [1..n-1]:
        j = i, t = a[j]
        while j > 0 && t < a[j-1]:
            a[j] = a[j-1]
            j = j-1
        end
        a[j] = t
    end
end
\end{verbatim}
В лучшем случае сравнений
$$
1 + 1 + . . . + 1 = n - 1
$$
в худшем случае
$$
1 + 2 + . . . + (n - 1) = \frac{n(n - 1)}{2}
$$


\section {Сортировка слиянием. Основная теорема для метода «разделяй и властвуй»}
\noindent{\bf Сортировка слиянием (Merge sort):}\\
В основе лежит идея «Разделяй и властвуй» (Divide and conquer):
\begin{enumerate}
\item Разделим массив на 2 части размера $\frac{n}{2}$
\item Отсортируем обе части (2 рекурсивных вызова)
\item Выполним процедуру слияния: объединяем отсортированные части таким образом, чтобы
получить полностью отсортированный массив
\end{enumerate}
Ключевая процедура - слияние (merge). Особенность: слияние требует дополнительный массив
\texttt{aux} (auxiliary - вспомогательный). Выделяем его сразу, чтобы не делать этого при
каждом вызове.
Код:
\begin{verbatim}
procedure MergeSort( a[] ):
    aux = new array [0..n-1]
    MergeSort(a, aux, 0, n-1)
end

procedure MergeSort( a[], aux[], low, high ):
    if (high <= low):
        return
    mid = low + (high - low) / 2
    MergeSort(a, aux, low, mid)
    MergeSort(a, aux, mid + 1, high)
    Merge(a, aux, low, mid, high)
end

procedure Merge( a[], aux[], low, mid, high ):
    for k in [low..high]:
        aux[k] = a[k]
    i = low, j = mid + 1
    for k in [low..high]:
        if i > mid:
            a[k] = aux[j++]
        else if j > high:
            a[k] = aux[i++]
        else if aux[j] < aux[i]:
            a[k] = aux[j++]
        else:
            a[k] = aux[i++]
    end
end
\end{verbatim}
Анализ сложности будет проведен позже.\\
{\bf Основная теорема (Master theorem):}\\
Пусть $n$ - размер задачи. $a$ - количество подзадач в рекурсии. $\frac{n}{b}$ - размер каждой подзадачи. $O(n^d)$ - оценка сложности работы, производимой алгоритмом вне рекурсивных вызовов. Пусть
$$
T(n) = aT(\left\lceil\frac{n}{b}\right\rceil) + O(n^d),\ \ \ a>0,\ b>1,\ d\ge0
$$
Тогда
$$
T(n) = 
\begin{cases}
 O(n^d),\ d>\log_{b}a \\
 O(n^d\log n),\ d=\log_{b}a \\
 O(n^{\log_{b}a}),\ d<\log_{b}a \\
\end{cases}
$$
Пользуясь этой теоремой, оценим сложность MergeSort. 
$$
T(n) = 2T(\frac{n}{2}) + O(n)
$$ 
В данном случае $a = 2,\ b = 2,\ d = 1$. Поэтому мы имеем дело со вторым случаем Основной теоремы, и $T(n) = O(n\log n)$.

\section {Алгоритм Карацубы умножения чисел, алгоритм Штрассена умножения матриц. Быстрое преобразование Фурье}

\noindent{\bf Алгоритм Карацубы} \\
Пусть есть два n-значных двоичных числа $A = a x + b$ и $B=c x + d$, где n — чётное число и $x=10^{\frac n 2}$. То есть, a и c получены из старших $\frac n 2$ разрядов A и B, а b и d — из младших. \\
\noindent В таких обозначениях произведение чисел A и B может быть переписано как
$$
AB = (a x + b)(c x + d) = a c x^2 + (a d + b c) x + b d
$$

\noindent Таким образом, умножение n-значных чисел было сведено к четырём задачам умножения $\frac n 2$-значных чисел и сложению результатов, которые выполняются за $O(n)$.
Далее можем заметить, что на самом деле достаточно лишь трёх умножений $\frac n 2$-значных чисел, так как
$$
(a d + b c) = (a + b)(c + d) - a c - b d
$$
Таким образом, всё произведение AB может быть получено из $a c$, $b d$ и $(a + b)(c + d)$ линейными операциями сложения, вычитания, а общее время работы оценивается как
$$
T(n) = 3 T (\frac {n}{2}) + O(n) =  O(n^{\log _{2}3})
$$

\noindent{\bf Алгоритм Штрассена} 

\noindent Хотим перемножить две матрицы. Основная идея: сводим умножение двух матриц размера $n$ к семи умножениям матриц размерности $\frac n 2$
\begin{equation*}
XY = \left(
\begin{array}{cc}
A & B \\
C & D\\
\end{array}
\right)
\left(
\begin{array}{cc}
E & F \\
G & H \\
\end{array}
\right)
=
\left(
\begin{array}{cc}
AE + BG & AF + BH \\
CE + DG & CF + DH \\
\end{array}
\right)
\end{equation*}

\noindent Можно заметить, что 
\begin{equation*}
XY = \left(
\begin{array}{cc}
P_5 + P_4 - P_2 + P_6 & P_1 + P_2 \\
P_3 + P_4 & P_1+ P_5 - P_3 - P_7\\
\end{array}
\right)
\end{equation*}

\begin{equation*}
\begin{array}{llllll}
P_1 &=& A(F - H) & P_2 &=& (A + B)H \\
P_3 &=& (C + D)E & P_4 &=& D(G - E) \\
P_5 &=& (A + D)(E + H) & P_6 &=& (B - D)(G + H) \\
P_7 &=& (A - C)(E + F) &  \\
\end{array}
\end{equation*}

\noindent В терминах основной теоремы из второго билета имеем (т.к. сложение матриц - это $n^2$):
$$
T(n) = 7T(\frac{n}{2}) + O(n^2)
$$
\noindent Поэтому сложность равна $O(n^{\log_2 7})$

\noindent{\bf Быстрое преобразование Фурье} \\
Теперь хотим перемножать многочлены. Пусть даны $A(x) = a_0 + a_1 x + .. + a_n x^n$ и $B(x) = b_0 + b_1 x + .. + b_n x^n$. Их произведение обозначим за $C(x).$ \\

\noindent Заметим, что если многочлены заданы не в виде набора коэффициентов, а в виде значений в (2n + 1) различных точках, то умножение многочленов работает за линейное время (2n + 1, так как многочлен-произведение имеет степень 2n). Действительно, пусть $A(p_i) = \alpha_i$, $B(p_i) = \beta_i$, тогда $C(p_i) = A(p_i) B(p_i) = \alpha_i * \beta_i$ \\

\noindent Теперь если мы научимся переходить от коэффициентов многочлена к значениям и, наоборот, по значениям вычислять коэффициенты за $O(n \log n)$, то итоговое умножение тоже будет работать за $O(n \log n)$ \\

\noindent Заметим, что
\begin{equation*}
\begin{array}{lll}
A(x) &=& A_0(x^2) + x A_1(x^2)\text{, где} \\
A_0(X) &=& a_0 + a_2 X + a_4 X^2 + ... \\
A_1(X) &=& a_1 + a_3 X + a_5 X^2 + ...
\end{array}
\end{equation*} \\

\noindent Так как в представлении A мы считаем $A_0$ и $A_1$ от $x^2$, то выбрав симметричные относительно нуля точки мы ускоримся в подсчете значений по коэффициентам в два раза (т.к при подсчете $A(x)$ и $A(-x)$ вычисляются одни и те же $A_0(x^2)$ и $A_1(x^2)$) \\

\noindent Но для применения основной теоремы из второго билета это сработает только один раз, и перейти от $\frac n 2$ к $\frac n 4$ уже не получится, поскольку все точки теперь положительные. Проблема решается выходом в комплексную плоскость и вычислением значений многочлена A(x) в корнях из единицы степени n (там каждый раз половина точек будет склеиваться и к концу как раз останется одна точка - единица) \\

\noindent Теперь рассмотрим переход от значений в корнях из единицы к коэффициентам многочлена (обратное преобразование Фурье). Давайте перепишем то, что мы сделали (переход от коэффициентов к значениям, оно же прямое преобразование Фурье) в матричном виде:

\begin{equation*}
\left(
\begin{array}{c}
A(1) \\
A(\omega) \\
A(\omega^2) \\
\vdots \\
A(\omega^n) 
\end{array}
\right)
= 
\left(
\begin{array}{cccc}
1 & 1 & \ldots & 1 \\
1 & \omega & \ldots & \omega^n \\
1 & \omega^2 & \ldots & \omega^{2n}\\
\vdots & \vdots & \ddots & \vdots\\
1 & \omega^{n} & \ldots & \omega^{n^2} \\
\end{array}
\right)
\left(
\begin{array}{c}
a_0 \\
a_1 \\
a_2 \\
\vdots \\
a_n 
\end{array}
\right)
= M_n (\omega)
\left(
\begin{array}{c}
a_0 \\
a_1 \\
a_2 \\
\vdots \\
a_n 
\end{array}
\right)
\end{equation*}
Чтобы перейти от значений к коэффициентам нужно умножить слева это на $M_n^{-1} (\omega)$. Непосредственным умножением проверяется, что $M_n^{-1} (\omega) = \frac{1}{n} M_n (-\omega)$. То есть в матричном виде принципиально ничего не поменялось (появилось умножение на $\frac{1}{n}$ и $-w$ вместо $w$, но это никак принципиально не влияет на тот алгоритм, который был показан). Поэтому этот же алгоритм применим и для получения коэффициентов по значениям.

\noindentВ терминах основной теоремы имеем (для преобразований Фурье):
$$
T(n) = 2T(\frac n 2) + O(n)
$$
То есть общая сложность $O(n \log n)$ \\

\noindent {\bf P.S.} Это все работает, когда n - степень двойки. Если это не так, то просто дополним старшие коэффициенты многочлена нулями до ближайшей степени двойки. Очевидно, что степень многочлена увеличится не более чем в два раза, что не влияет на ассимптотику, т.к $O(2 n \log(2 n)) = O(n \log n)$

\section {Оценка снизу количества сравнений при сортировке, быстрая сортировка Хоара, сложность в среднем и в худшем. Задача Дейкстры о голландском флаге, 3х-частное разбиение, эвристики выбора опорного элемента} 

\noindent {\bf Теорема.} Любой детерменированный алгоритм сортировки сравнением имеет в худшем случае $\Omega (n log n)$ \\

\noindent Количество перестановок в массиве из n элементов $= n!$ \\
Работу алгоритма на различных входных данных можно
представить в виде бинарного дерева. Каждое ветвление –
сравнение элементов массива. Если при любых входных данных
количество сравнений не больше S, то глубина дерева не больше
S.\\

\noindent Оценим теперь глубину этого дерева. По определению глубины дерева, в дереве глубины S кол-во путей $\le 2^S$. C другой стороны, так как сортировка должна работать для любого массива, то итоговое кол-во путей должно быть $\ge n!$. Поэтому $2^S \ge n!$, то есть $s \ge log_2 n! \ge log_2 \frac{n}{2}^{\frac{n}{2}} \ge k n log n$ для некоторого k. \\

\noindent{\bf Быстрая сортировка:}

\noindent Алгоритм состоит из трёх шагов:
\begin{enumerate}
    \item Выбрать элемент из массива. Назовём его опорным
    \item Разбиение: перераспределение элементов в массиве таким образом, что элементы меньше опорного помещаются перед ним, а больше или равные после
    \item Рекурсивно применить первые два шага к двум подмассивам слева и справа от опорного элемента. Рекурсия не применяется к массиву, в котором только один элемент или отсутствуют элементы
\end{enumerate}

\begin{verbatim}
procedure QuickSort(a[], left, right):
    // a[] - массив от 0 до n-1
    pivot = a[left]
    i = left + 1
    for j in [left + 1 .. right]:
        if a[j] < pivot:
            переставить a[j] и a[i]
            i++
        end
    end
    переставить a[left] и a[i-1]
    
    QuickSort(a, left, i - 2)
    QuickSort(a, i, right)
end
\end{verbatim} \\

\noindent Сложность
\begin{itemize}
    \item В худшем случае: $O(n^2)$
    \item В лучшем случае (опорный элемент - медиана):
    $$
    T(n) = 2T(\frac{n}{2})+ O(n) \Rightarrow T(n) = O(n log n)
    $$
    \item В «плохом» случае - опорный элемент делит массив в соотношении 99/100: $O(n log n)$
\end{itemize}

\noindent{\bf Теорема.} Сложность быстрой сортировки «в среднем» (при случайном выборе опорного элемента) равно $O (n log n)$

\noindent{\bf Доказательство:}
Пусть пространство элементарных событий $\Omega =$ \{все возможные последовательности опорных элементов\}, cлучайная величина $X(\omega) =$ \{количество выполняемых сравнений при последовательности опорных элементов $\omega$\}. Наша цель: мат. ожидание $M[X] = O (n log n)$

\noindent Пусть $z_i$ - i-я порядковая статистика (стоит на i-м месте в отсортированном массиве), $X_{ij} (\omega)$ - кол-во сравнений $z_i$ и $z_j$ (заметим, что сравнений всегда либо 0, либо 1 вне зависимости от выборов опорных элементов). \\ 
\begin{equation*}
\begin{array}{c}
X(\omega) = \sum\limits_{i<j} X_{ij}(\omega) \\
M[X] = \sum\limits_{i<j} M[X_{ij}] = \sum\limits_{i<j} P[X_{ij} = 1]
\end{array}
\end{equation*}
$P[X_{ij} = 1] = \frac{2}{j - i + 1}$, так как рассмотрим все элементы в отсортированным массиве, стоящие с позиции i до позиции j (их $j - i + 1$ штук). В зависимости от того какой из этих элементов первый будет опорным возможны случаи. Если опорный будет i-й или j-й, то $X_{ij} = 1$, иначе эти два элемента попадут по разные стороны от опорного и их сравнение не произойдет, то есть $X_{ij} = 0$. Поэтому:
\begin{equation*}
\begin{array}{c}
M[X] = \sum\limits_{i<j} P[X_{ij} = 1] = \sum\limits_{i<j} \frac{2}{j - i + 1} = \\ 
= \sum\limits_{i=1}^{n-1} \sum\limits_{j=i+1}^{n} \frac{2}{j - i + 1}
\le 2n \sum\limits_{k=2}^{n} \frac 1 k \le 2nlnn
\end{array}

\noindent Свойства:
\begin{itemize}
    \item Требует $\approx log n$ дополнительной памяти
    \item Не является устойчивым (может не сохранять порядок одинаковых элементов)
\end{itemize}

\noindent{\bf Трехчастное разбиение:} 

\noindent Заметим, что сложность быстрой сортировки на массиве из одинаковых элементов $= O(n^2)$. Решение - трехчастное разбиение:
\begin{itemize}
    \item a[i] < p для $0 \le i\le r$ - меньше опорного
    \item a[i] = p для $r + 1 \le i \le s$ - равны опорному
    \item a[i] > p для $s + 1 \le i \le n - 1$ - больше опорного
\end{itemize}

\begin{verbatim}
procedure QuickSort3Way(a[]):
    // a[] - массив от 0 до n-1
    перемешать (shuffle) массив a
    QuickSort3Way(a, 0, n-1)
end

procedure QuickSort3Way(a[], low, high):
    p = a[low]
    i = low + 1
    lt = low + 1
    gt = high
    while i <= gt:
        if a[i] < p:
            Exch(a, lt++, i++)
        else if a[i] > p:
            Exch(a, i, gt--)
        else
            i++
    end
    Exch(a, low, --lt)
    QuickSort3Way(a, low, lt - 1)
    QuickSort3Way(a, gt + 1, high)
end
\end{verbatim}

\noindent{\bf Задача о голландском флаге:} отсортировать массив, состоящий только из нулей, единиц и двоек. Как раз для этой задачи трехчастное разбиение идеально подходит.

\noindent{\bf Практические улучшения aka эвристики:}

\begin{enumerate}
    \item На массивах небольшого размера выгоднее использовать сортировку вставками
    \item Выбор опорного элемента: медиана из трех (median-of-3). Выбираем 3 случайны элемента и в качестве опорного берем второй по порядку (медиану).
    \item Выбор опорного элемента: выбираем 3 тройки случайных элементов, в
    каждой тройке находим медиану и в качестве опорного берем медиану медиан.
\end{enumerate}


\section*{10. Хеш-таблицы, реализация методом цепочек и открытой адресацией.}

\subsection*{Хэш-таблицы}

{\bf Ассоциативный массив} - структура данных, хранящая наборы типа <Key, Value>. 
Предполагается, что ассоциативный массив не может хранить две пары с одинаковыми ключами (на практике этого добиться сложно).
Интерпретация ассоциативного массива - отображение $value: Key \rightarrow Value$.

Операции (или интерфейс):
\begin{itemize}
\item Вставка (добавление)
\item Поиск по ключу (взятие значения или проверка)
\item Удаление по ключу
\end{itemize}

{\bf Хеш-таблица} - структура, реализующая ассоциативный массив.

{\bf Важные свойства} хеш-таблиц состоит в том, что, при некоторых разумных допущениях, в среднем:
\begin{itemize}
\item Вставка: O(1)
\item Поиск по ключу: O(1)
\item Удаление по ключу: O(1)
\item Память: O(n)
\end{itemize}

Отображение $h: U \rightarrow \{0, 1, ..., k\}$, где $U$ - множество ключей называется {\bf хэш-функций}.
Если $h$ - хэш-функция, то можно считать, что $value(u) = array[h(u)]$, где $array$ - массив размера $m$.

Пример:
$"abc" \rightarrow ord(a)*ord(b)*ord(c)$.

Код:
\begin{verbatim}
public int hash(char[] value) 
{
	int h = 0;
	for (int i = 0; i < value.length; i++) 
	{
		h = 31 * h + value[i];
	}
	return h;
}
\end{verbatim}

В таком определении могут возникнуть {\bf коллизии} $h(u_1) = h(u_2)$, где $u_1, u_2 \in U$.

\subsection*{Разрешение коллизий. Метод цыпочек}
Вычисляем $h(u)$. В случае коллизии пихаем элемент в "соседнюю"(это понятие резиновое, например правая или вторая справа. Вообщем какая-то функция от положения в array.) ячейку.
Как искать? - пробегаем все соседнии ячейки, пока ключ совпадает.

\subsubsection*{Теорема}
Математическое ожидание сложности неудачного поиска
при условии простого равномерного хеширования равна
$O(1 + \alpha)$, где $\alpha$ - коэффициент заполнения таблицы.
$\alpha = \frac{n}{m}$, где $m$ - число ячеек. 

\subsection*{Разрешение коллизий. Открытой адресации}
$array[h(u)]$ - список. В случае коллизий append-им пару (ключ, значение) туда.


\subsubsection*{Теорема}
Математическое ожидание сложности неудачного поиска
при условии простого равномерного хеширования равна
$O(\frac{1}{1-\alpha})$, где $\alpha < 1$ - коэффициент заполнения таблицы.
$\alpha = \frac{n}{m}$, где $m$ - число ячеек. 

Доказательство:

Пусть $X$ - количество проверок при неуспешном поиске.
$$E[X] = \sum\limits_{i=0}^{\infty}iP\{X = i\} = \sum\limits_{i=0}^{\infty}iP(\{X\ge i\} - P\{X\ge i+1\}) = \sum\limits_{i=0}^{\infty}P\{X \ge i\}$$
$$E[X] = \sum\limits_{i=1}^{\infty}P{X \ge i}\ge\sum\limits_{i=0}^{\infty}\alpha^i = \frac{1}{1-\alpha}$$ 

\subsection*{Защита хэш-функций от атак}
Если недоброжелателю известна хэш-функция $h$, то он может нагенерировать очень много коллизий, что приведет к работе алгоритма за $O(n)$ (так называемая $атака на сервер.$)
{\bf Универсальным хэшированием} называется решение такой проблемы - просто выбираем случайную хэш-функцию из какого-то множества.

\section{Распределенные хеш-таблицы. Фильтр Блума.}
 

 {\bf Распределенной хеш-таблицей} называется такая таблица, которая распределена на разные узлы (сервера).
 Имеется проблема: когда отказывает узел, то в обычной таблице перестраивается все. Как устранить эту проблему в распределенной таблице?

\subsection {Консистентное (согласованное) хеширование}
Берем наши обьекты и распределяем их по единичному кругу (согласно хеш-функции). Вклиниваем равномерно наши сервера между этими обьектами. Идем по часой стрелки от перого обьекта. Первый встречный узел - владелец этого обьекта. В таком случае, при отключении или добавлении нового узла перестраивается лишь небольшая часть таблицы. 

\begin{table}[]
\begin{tabular}{|c|c|c|}
\hline
\multicolumn{3}{|c|}{m - размер таблицы, N - число узлов}       \\ \hline
              & Обычная хеш-таблица & Согласованное хеширование \\ \hline
Добавить узел & O(m)                & O(m/N + logN)             \\ \hline
Удалить узел  & O(m)                & O(m/N + logN)             \\ \hline
Добавить ключ & O(1)                & O(logN)                   \\ \hline
Удалить ключ  & O(1)                & O(logN)                   \\ \hline
\end{tabular}
\end{table}

\subsection {Рандеву хеширование}
Рассматриваем следующую хеш-функцию: $h: U \times S \rightarrow {0, ..., m}$. S - множество всех узлов. Изначально разбрасываем обьекты на узлы с максимальным весом, т.е. 
$u \rightarrow S_{dst}$, где $h(u, S_{dst}) = \min h(u, S_i)$. Очевидно, если узел был удален, или добавлен, то почти ничего не меняется.


\subsection {Фильтр Блума}
Пусть имеется $m$ бит (фильтр, mask) и $k$ хеш-функций. Считаем, что хеш-функции - независимые случайные величины, такие, что 
$$P(h_s(u) = i) = \frac{1}{m}$$

\begin{itemize}
\item Вставка: Записываем $mask[h_s(u)] = 1$, $\forall s$.
\item Поиск по ключу: Если $mask[h_s(u)] == 1$, $\forall s$, то элемент присутствует в хеш-таблице.
\end{itemize}

Вероятность того, что в $i$-ый бит не будет записана единица во время вставки элемента есть 
$$P(h_1(u) \ne i)...P(h_s(u) \ne i)...P(h_k(u) \ne i) = \left( 1 - \frac{1}{m}\right)^k$$

Вероятность того, что $i$-ый бит останется равным нулю после вставки $n$ различных элементов в изначально пустой фильтр Блума есть 
$$\left( 1 - \frac{1}{m}\right)^{nk} \approx e^{-\frac{kn}{m}}$$

Ложноположительное срабатывание состоит в том, что для некоторого элемента $u$, не равного ни одному из вставленных, все $k$ бит с номерами $h_s(u)$ окажутся ненулевыми, и фильтр Блума ошибочно ответит, что $u$ входит во множество вставленных элементов. Вероятность такого события примерно равна
$$(1 - e^{-\frac{kn}{m}})^k$$

Эта вероятность падает с ростом $m$ (размер фильтра) и растет с ростом $n$ (число вставленных элементов).


\section{Топологическая сортировка. Алгоритм Косарайю поиска сильносвязанных компонент.}
\noindent{\bf Топологическая сортировка:}\\
Дано: Ориентированный ациклический граф $G = (V, E)$.\\
Надо: Пронумеровать вершины $V$ графа так, чтобы для любого ребра $(u, v) \in E$ номер вершины $v$ был больше номера вершины $u$.\\
(Другими словами: расположить вершины на временной оси так, чтобы все ребра были направлены "в будущее".)\\
Решение: Выводим вершины в обратном post-порядке.\\
Через $A(v),\ v\in V$ обозначим множество таких вершин $u\in V$, что $\exists\  (u,v)\in E$.
То есть, $A(v)$ - множество всех вершин, из которых есть ребро в вершину $v$. Пусть $P$ -
искомая последовательность вершин.\\
Псевдокод:
\begin{verbatim}
while |P|<|V|:
    выбрать любую вершину v такую, что A(v) 
                пусто, и v не принадлежит P
    P.append (v)
    удалить v из всех A(u), где u != v
end
\end{verbatim}
\note{Пример работы алгоритма есть в Википедии:\\
https://ru.wikipedia.org/wiki/Топологическая\_сортировка}\\

На компьютере топологическую сортировку можно выполнить за $O(n)$ времени и памяти,
если обойти все вершины, используя DFN, и выводить вершины в момент выхода из неё.\\
Другими словами, алгоритм (Тарьяна) состоит в следующем:
\begin{itemize}
\item Изначально все вершины белые.
\item Для каждой вершины делаем шаг алгоритма.
\end{itemize}
Шаг алгоритма:
\begin{itemize}
\item Если вершина чёрная, ничего делать не надо.
\item Если вершина серая - найден цикл, топологическая сортировка невозможна.
\item Если вершина белая
\begin{itemize}
\item Красим её в серый
\item Применяем шаг алгоритма для всех вершин, в которые можно попасть из текущей
\item Красим вершину в чёрный и помещаем её в начало окончательного списка.
\end{itemize}
\end{itemize}

\noindent{\bf Алгоритм Косарайю:}\\
Ориентированный граф называется {\bf сильно связным (strongly connected)}, если любые две его вершины $s$ и $t$ сильно связны, т.е. если существует ориентированный путь из $s$ в $t$ и ориентированный путь из $t$ в $s$. Компонентами сильной связности орграфа называются его максимальные по включению сильно связные подграфы.\\
{\bf Компонента-сток} - (простыми словами) компонента, в которую можно только войти, но нельзя из нее выйти.\\
{\it Утв 1.} Пусть $v$ лежит в компоненте-стоке. Тогда DFS($G$, $v$) обойдет все вершины в компоненте и только их.\\
{\it Утв 2.} Пусть $C$ и $C'$ - компоненты сильной связности, и существует ребро из $C$ в $C'$. Тогда
$$
\max\limits_{c\in C} post(c) > \max\limits_{c'\in C'} post(c')
$$
{\it Следствие.} Вершина с максимальным post-значением лежит в компоненте-истоке.\\

\noindentАлгоритм Косарайю:
\begin{enumerate}
\item Инвертируем ребра (меняем их направление) исходного ориентированного графа - получаем обращённый граф.
\item Запускаем DFS на этом обращённом графе, запоминая, в каком порядке выходили из вершин.
\item Запускаем DFS на исходном графе, в очередной раз выбирая не посещённую вершину с максимальным номером в векторе, полученном в п.2.
\item Полученные из п.3 деревья и являются сильно связными компонентами (Каждое множество вершин, достигнутое в результате очередного запуска обхода, и будет очередной компонентой сильной связности.)
\end{enumerate}
Подробное и понятное описание алгоритма по ссылке:\\
https://e-maxx.ru/algo/strong\_connected\_components\\
Сложность: $O(n)$. 

\section{Очередь с приоритетами, реализация с помощью бинарной кучи. Пирамидальная сортировка. Алгоритм Дейкстры.}
\noindent{\bf Очередь с приоритетами:}\\
Очередь с приоритетом (priority queue) - абстрактный тип данных, поддерживающий две обязательные операции - добавить элемент и извлечь максимум (минимум). Предполагается, что для каждого элемента можно вычислить его {\it приоритет} - действительное число или в общем случае элемент линейно упорядоченного множества.\\
Операции (интерфейс):
\begin{itemize}
\item Добавить элемент (insert)
\item Исключить минимальный элемент (deleteMin)
\end{itemize}
Дополнительные операции (интерфейс):
\begin{itemize}
\item Построить очередь (makeQueue / buildHeap)
\item Уменьшить значение ключа (decreaseKey)
\end{itemize}
Варианты реализации:
\begin{enumerate}
\item Неупорядоченный массив.\\
insert: $O(1)$ - просто добавляем в конец\\
deleteMin: $O(n)$ - надо перебрать все элементы
\item Упорядоченный массив (по убыванию).\\
insert: $O(n)$ - приходится сдвигать кусок массива\\
deleteMin: $O(1)$ - просто берем последний элемент
\item {\bf Бинарная куча (Heap, Binary heap).}\\
insert: $O(\log n)$\\
deleteMin: $O(\log n)$\\
buildHeap: $O(n)$\\
decreaseKey: $O(\log n)$
\end{enumerate}
Если кратко, бинарная куча - массив следующего вида:\\
для элемента с индексом $i$: индексы "детей"{} = \{$2i$, $2i + 1$\}, индекс "родителя"{} = $i/2$.\\
Бинарную кучу следует представлять себе как дерево, для которого выполнены три условия:
\begin{itemize}
\item Значение в любой вершине не меньше, чем значения её потомков.
\item Глубина всех листьев (расстояние до корня) отличается не более чем на 1 слой.
\item Последний слой заполняется слева направо без «дырок».
\end{itemize}
У бинарной кучи существуют операции swim (всплытие узла) и sink (погружение узла).\\
Код:
\begin{verbatim}
h = array of keys of length N

procedure swim(i):
    while i > 1 and h[i/2] < h[i]:
        h[i/2], h[i] = h[i], h[i/2]
        i = i / 2
    end
end

procedure sink(i):
    while 2*i <= N:
        int k = argmin(h[2*i], h[2*i + 1])
        if h[k] < h[i]:
            h[k], h[i] = h[i], h[k]
        else:
            break
    end
end
\end{verbatim}
Построение кучи, которое описано ниже, соответствует операции makeQueue и выполняется за $O(n)$:
\begin{verbatim}
N = array length
h = array of keys of length N

procedure buildHeap():
    for i = N to 1:
        sink(i)
end
\end{verbatim}
Подробнее и более понятно о бинарной куче: \href{https://clck.ru/ScTWD}{Двоичная куча}\\

\noindent{\bf Heapsort (пирамидальная сортировка):}\\
Процедура Heapsort сортирует массив (бинарную кучу) без привлечения дополнительной памяти за
$O(n \log n)$.\\
Будем удалять элементы из корня по одному за раз и перестраивать дерево. То есть на первом шаге обмениваем \texttt{h[0]} и \texttt{h[N-1]}, преобразовываем \texttt{h[0], h[1], ... , h[N-2]} в бинарную кучу. Затем переставляем \texttt{h[0]} и \texttt{h[N-2]}, преобразовываем \texttt{h[0], h[1], ... , h[N-3]} в бинарную кучу и т.д. Процесс продолжается до тех пор, пока в куче не останется один элемент. Тогда \texttt{h[0], h[1], ... , h[N-1]} - упорядоченная последовательность.\\
\note{Нужно, наверное, написать подробнее про Heapsort, но на лекциях этого вообще толком не было...}\\
{\bf Алгоритм Дейкстры:}\\
Пусть на ребрах графа заданы длины: $G = (V, E),\ l : E \rightarrow \mathbf{R}_{+}$\\
Задача: найти кратчайшие расстояния до всех вершин из данной вершины $s$.\\
Код:
\begin{verbatim}
procedure Dijkstra( G, s ):
    dist = array of size G.V
    prev = array of size G.V

    for x in G.V:
        dist[x] = inf
        prev[x] = -1

    dist[s] = 0
    Q = priority queue
    Q.makeQueue(V, dist)

    while not Q.isEmpty:
        x = Q.deleteMin()
        for y in G.neighbours(x):
            key = dist[x] + l(x, y)
            if dist[y] > key:
                dist[y] = key
                prev[y] = x
                Q.decreaseKey(y, key)
            end
        end
    end
end
\end{verbatim}
Подробнее: \href{https://ru.wikipedia.org/wiki/Алгоритм_Дейкстры}{Википедия}

 
\end{document}
