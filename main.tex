%main.tex
\documentclass[specialist, subf, href, colorlinks=true, 12pt, times, mtpro, final]{disser}
\usepackage[utf8]{inputenc}
\usepackage[english,russian]{babel}
\usepackage{a4wide}
\usepackage{mathtext}
\usepackage{amsbsy}
\usepackage{amsthm}
\usepackage{amsmath}
\usepackage{amssymb}
\usepackage{amsfonts}
\usepackage{tikz}
\usepackage{verbatim}
\usepackage{graphicx}

\definecolor{faded}{gray}{0.6}
\def\note{\textcolor{faded}}
 
\begin{document}

\tableofcontents

\newpage
{\Large \bf Программа экзамена (осень 2020)}

\begin{enumerate}
{\footnotesize
\item Сложность алгоритмов, бинарный поиск, сортировка выбором и вставками
\item Сортировка слиянием. Основная теорема для метода «разделяй и властвуй».
\item Алгоритм Карацубы умножения чисел, алгоритм Штрассена умножения матриц. Быстрое преобразование Фурье.
\item Оценка снизу количества сравнений при сортировке, быстрая сортировка Хоара, сложность в среднем и в худшем. Задача Дейкстры о голландском флаге, 3х-частное разбиение, эвристики выбора опорного элемента.
\item Алгоритмы нахождения k-й порядковой статистики - вероятностный и детерминированный. Метод сортировки TimSort.
\item Абстрактные типы данных, интерфейс и реализация, стек и очередь - реализация связанным списком и массивом.
\item Ассоциативные массивы, бинарные деревья поиска.
\item Сбалансированные деревья, 2-3 и красно-черные деревья.
\item B-деревья.
\item Хеш-таблицы, реализация методом цепочек и открытой адресацией.
\item Распределенные хеш-таблицы. Фильтр Блума.
\item Графы, способы представления в программе. Поиск в глубину и поиск в ширину.
\item Топологическая сортировка. Алгоритм Косарайю поиска сильносвязанных компонент.
\item Очередь с приоритетами, реализация с помощью бинарной кучи. Пирамидальная сортировка. Алгоритм Дейкстры.
}
\end{enumerate}

\section {Сложность алгоритмов, бинарный поиск, сортировка выбором и вставками}
\noindentПусть $\Omega_{n}$ – множество всех допустимых наборов входных данных размера $n$.\\
$T(\omega)$ – {\bf сложность} (количество операций) алгоритма $A$ на входных данных $\omega \in \Omega_{n}$.\\
Сложность {\bf в худшем случае}:
$$
  T(n) = \max\limits_{\omega \in \Omega_{n}}T(\omega)
$$
Сложность {\bf в среднем (average)}:
$$
  T_{avg}(n) = \frac{1}{|\Omega_{n}|}\sum\limits_{\omega \in \Omega_{n}}T(\omega)
$$
\note{// Отступление про $O$-нотацию.}\\
{\bf Поиск элемента в массиве (бинарный поиск):}\\
Дано: отсортированный массив \texttt{a[]}, элемент \texttt{x}\\
Требуется: определить, содержится ли \texttt{x} в \texttt{a[]}\\
Идея:
\begin{enumerate}
\item Делим массив пополам
\item Сравниваем \texttt{x} с центральным элементом \texttt{a[mid]}
\begin{itemize}
\item \texttt{x = a[mid]}? – нашли
\item \texttt{x < a[mid]}? – ищем в нижней части массива
\item \texttt{x > a[mid]}? – ищем в верхней части массива
\end{itemize}
\end{enumerate}
Код:\\
\begin{verbatim}
procedure BinarySearch( a[], x ):
    // a[] - array from 0 to n-1
    low = 0
    high = n-1
    while low <= high:
        mid = low + (high - low) / 2
        if a[mid] < x:
            low = mid + 1
        else if a[mid] > x:
            high = mid - 1
        else
            return mid
    end
    return -1
end
\end{verbatim}
Пусть $n \le 2^k$ ($k = [\log n]$). Итераций: $k + 1$. На каждой итерации: $O(1)$. Итого:
$O(\log n)$.\\
{\bf Сортировка выбором (Selection sort):}\\
Шаги алгоритма:
\begin{enumerate}
\item Находим номер минимального (или максимального, смотря что хотим) значения в текущем списке
\item Производим обмен этого значения со значением первой неотсортированной позиции (обмен не нужен, если минимальный элемент уже находится на данной позиции)
\item Теперь сортируем хвост списка, исключив из рассмотрения уже отсортированные элемент
\end{enumerate}
Код:
\begin{verbatim}
procedure SelectionSort( a[] ):
    // a[] - array from 0 to n-1
    for i in [1..n-1]:
        j = index of max element in a[0..n-i]
    swap (a[j], a[n-i])
    end
end
\end{verbatim}
Количество сравнений:
$$
(n-1)+(n-2)+...+1 = \frac{n(n-1)}{2}
$$
Итого: $O(n^2)$.\\
{\bf Сортировка вставками (Insertion sort):}\\
Идея: для каждого элемента ищем место в отсортированной части массива, в которое его нужно вставить.\\
Код:
\begin{verbatim}
procedure InsertionSort( a[] ):
    for i in [1..n-1]:
        j = i, t = a[j]
        while j > 0 && t < a[j-1]:
            a[j] = a[j-1]
            j = j-1
        end
        a[j] = t
    end
end
\end{verbatim}
В лучшем случае сравнений
$$
1 + 1 + . . . + 1 = n - 1
$$
в худшем случае
$$
1 + 2 + . . . + (n - 1) = \frac{n(n - 1)}{2}
$$


\section {Сортировка слиянием. Основная теорема для метода «разделяй и властвуй»}
\noindent{\bf Сортировка слиянием (Merge sort):}\\
В основе лежит идея «Разделяй и властвуй» (Divide and conquer):
\begin{enumerate}
\item Разделим массив на 2 части размера $\frac{n}{2}$
\item Отсортируем обе части (2 рекурсивных вызова)
\item Выполним процедуру слияния: объединяем отсортированные части таким образом, чтобы
получить полностью отсортированный массив
\end{enumerate}
Ключевая процедура - слияние (merge). Особенность: слияние требует дополнительный массив
\texttt{aux} (auxiliary - вспомогательный). Выделяем его сразу, чтобы не делать этого при
каждом вызове.
Код:
\begin{verbatim}
procedure MergeSort( a[] ):
    aux = new array [0..n-1]
    MergeSort(a, aux, 0, n-1)
end

procedure MergeSort( a[], aux[], low, high ):
    if (high <= low):
        return
    mid = low + (high - low) / 2
    MergeSort(a, aux, low, mid)
    MergeSort(a, aux, mid + 1, high)
    Merge(a, aux, low, mid, high)
end

procedure Merge( a[], aux[], low, mid, high ):
    for k in [low..high]:
        aux[k] = a[k]
    i = low, j = mid + 1
    for k in [low..high]:
        if i > mid:
            a[k] = aux[j++]
        else if j > high:
            a[k] = aux[i++]
        else if aux[j] < aux[i]:
            a[k] = aux[j++]
        else:
            a[k] = aux[i++]
    end
end
\end{verbatim}
Анализ сложности будет проведен позже.\\
{\bf Основная теорема (Master theorem):}\\
Пусть $n$ - размер задачи. $a$ - количество подзадач в рекурсии. $\frac{n}{b}$ - размер каждой подзадачи. $O(n^d)$ - оценка сложности работы, производимой алгоритмом вне рекурсивных вызовов. Пусть
$$
T(n) = aT(\left\lceil\frac{n}{b}\right\rceil) + O(n^d),\ \ \ a>0,\ b>1,\ d\ge0
$$
Тогда
$$
T(n) = 
\begin{cases}
 O(n^d),\ d>\log_{b}a \\
 O(n^d\log n),\ d=\log_{b}a \\
 O(n^{\log_{b}a}),\ d<\log_{b}a \\
\end{cases}
$$
Пользуясь этой теоремой, оценим сложность MergeSort. 
$$
T(n) = 2T(\frac{n}{2}) + O(n)
$$ 
В данном случае $a = 2,\ b = 2,\ d = 1$. Поэтому мы имеем дело со вторым случаем Основной теоремы, и $T(n) = O(n\log n)$.


\section*{10. Хеш-таблицы, реализация методом цепочек и открытой адресацией.}

\subsection*{Хэш-таблицы}

{\bf Ассоциативный массив} - структура данных, хранящая наборы типа <Key, Value>. 
Предполагается, что ассоциативный массив не может хранить две пары с одинаковыми ключами (на практике этого добиться сложно).
Интерпретация ассоциативного массива - отображение $value: Key \rightarrow Value$.

Операции (или интерфейс):
\begin{itemize}
\item Вставка (добавление)
\item Поиск по ключу (взятие значения или проверка)
\item Удаление по ключу
\end{itemize}

{\bf Хеш-таблица} - структура, реализующая ассоциативный массив.

{\bf Важные свойства} хеш-таблиц состоит в том, что, при некоторых разумных допущениях, в среднем:
\begin{itemize}
\item Вставка: O(1)
\item Поиск по ключу: O(1)
\item Удаление по ключу: O(1)
\item Память: O(n)
\end{itemize}

Отображение $h: U \rightarrow \{0, 1, ..., k\}$, где $U$ - множество ключей называется {\bf хэш-функций}.
Если $h$ - хэш-функция, то можно считать, что $value(u) = array[h(u)]$, где $array$ - массив размера $m$.

Пример:
$"abc" \rightarrow ord(a)*ord(b)*ord(c)$.

Код:
\begin{verbatim}
public int hash(char[] value) 
{
	int h = 0;
	for (int i = 0; i < value.length; i++) 
	{
		h = 31 * h + value[i];
	}
	return h;
}
\end{verbatim}

В таком определении могут возникнуть {\bf коллизии} $h(u_1) = h(u_2)$, где $u_1, u_2 \in U$.

\subsection*{Разрешение коллизий. Метод цыпочек}
Вычисляем $h(u)$. В случае коллизии пихаем элемент в "соседнюю"(это понятие резиновое, например правая или вторая справа. Вообщем какая-то функция от положения в array.) ячейку.
Как искать? - пробегаем все соседнии ячейки, пока ключ совпадает.

\subsubsection*{Теорема}
Математическое ожидание сложности неудачного поиска
при условии простого равномерного хеширования равна
$O(1 + \alpha)$, где $\alpha$ - коэффициент заполнения таблицы.
$\alpha = \frac{n}{m}$, где $m$ - число ячеек. 

\subsection*{Разрешение коллизий. Открытой адресации}
$array[h(u)]$ - список. В случае коллизий append-им пару (ключ, значение) туда.


\subsubsection*{Теорема}
Математическое ожидание сложности неудачного поиска
при условии простого равномерного хеширования равна
$O(\frac{1}{1-\alpha})$, где $\alpha < 1$ - коэффициент заполнения таблицы.
$\alpha = \frac{n}{m}$, где $m$ - число ячеек. 

Доказательство:

Пусть $X$ - количество проверок при неуспешном поиске.
$$E[X] = \sum\limits_{i=0}^{\infty}iP\{X = i\} = \sum\limits_{i=0}^{\infty}iP(\{X\ge i\} - P\{X\ge i+1\}) = \sum\limits_{i=0}^{\infty}P\{X \ge i\}$$
$$E[X] = \sum\limits_{i=1}^{\infty}P{X \ge i}\ge\sum\limits_{i=0}^{\infty}\alpha^i = \frac{1}{1-\alpha}$$ 

\subsection*{Защита хэш-функций от атак}
Если недоброжелателю известна хэш-функция $h$, то он может нагенерировать очень много коллизий, что приведет к работе алгоритма за $O(n)$ (так называемая $атака на сервер.$)
{\bf Универсальным хэшированием} называется решение такой проблемы - просто выбираем случайную хэш-функцию из какого-то множества.

 
\end{document}
