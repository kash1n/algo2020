%main.tex
\documentclass[specialist, subf, href, colorlinks=true, 12pt, times, mtpro, final]{disser}
\usepackage[utf8]{inputenc}
\usepackage[english,russian]{babel}
\usepackage{a4wide}
\usepackage{mathtext}
\usepackage{amsbsy}
\usepackage{amsthm}
\usepackage{amsmath}
\usepackage{amssymb}
\usepackage{amsfonts}
\usepackage{tikz}
\usepackage{verbatim}
\usepackage{graphicx}
\usepackage{xcolor}
\usepackage{hyperref}
 
 % Цвета для гиперссылок
\definecolor{linkcolor}{HTML}{543AE8} % цвет ссылок
\definecolor{urlcolor}{HTML}{543AE8} % цвет гиперссылок
\hypersetup{pdfstartview=FitH,  linkcolor=linkcolor,urlcolor=urlcolor, colorlinks=true}
 
 % Цвет для комментариев вида \note{...}
\definecolor{faded}{gray}{0.6}
\def\note{\textcolor{faded}}
 
\begin{document}

\tableofcontents

\newpage
{\Large \bf Программа экзамена (осень 2020)}

\begin{enumerate}
{\footnotesize
\item Сложность алгоритмов, бинарный поиск, сортировка выбором и вставками
\item Сортировка слиянием. Основная теорема для метода «разделяй и властвуй».
\item Алгоритм Карацубы умножения чисел, алгоритм Штрассена умножения матриц. Быстрое преобразование Фурье.
\item Оценка снизу количества сравнений при сортировке, быстрая сортировка Хоара, сложность в среднем и в худшем. Задача Дейкстры о голландском флаге, 3х-частное разбиение, эвристики выбора опорного элемента.
\item Алгоритмы нахождения k-й порядковой статистики - вероятностный и детерминированный. Метод сортировки TimSort.
\item Абстрактные типы данных, интерфейс и реализация, стек и очередь - реализация связанным списком и массивом.
\item Ассоциативные массивы, бинарные деревья поиска.
\item Сбалансированные деревья, 2-3 и красно-черные деревья.
\item B-деревья.
\item Хеш-таблицы, реализация методом цепочек и открытой адресацией.
\item Распределенные хеш-таблицы. Фильтр Блума.
\item Графы, способы представления в программе. Поиск в глубину и поиск в ширину.
\item Топологическая сортировка. Алгоритм Косарайю поиска сильносвязанных компонент.
\item Очередь с приоритетами, реализация с помощью бинарной кучи. Пирамидальная сортировка. Алгоритм Дейкстры.
}
\end{enumerate}

\section {Сложность алгоритмов, бинарный поиск, сортировка выбором и вставками}
\noindentПусть $\Omega_{n}$ – множество всех допустимых наборов входных данных размера $n$.\\
$T(\omega)$ – {\bf сложность} (количество операций) алгоритма $A$ на входных данных $\omega \in \Omega_{n}$.\\
Сложность {\bf в худшем случае}:
$$
  T(n) = \max\limits_{\omega \in \Omega_{n}}T(\omega)
$$
Сложность {\bf в среднем (average)}:
$$
  T_{avg}(n) = \frac{1}{|\Omega_{n}|}\sum\limits_{\omega \in \Omega_{n}}T(\omega)
$$
\note{// Отступление про $O$-нотацию.}\\
{\bf Поиск элемента в массиве (бинарный поиск):}\\
Дано: отсортированный массив \texttt{a[]}, элемент \texttt{x}\\
Требуется: определить, содержится ли \texttt{x} в \texttt{a[]}\\
Идея:
\begin{enumerate}
\item Делим массив пополам
\item Сравниваем \texttt{x} с центральным элементом \texttt{a[mid]}
\begin{itemize}
\item \texttt{x = a[mid]}? – нашли
\item \texttt{x < a[mid]}? – ищем в нижней части массива
\item \texttt{x > a[mid]}? – ищем в верхней части массива
\end{itemize}
\end{enumerate}
Код:\\
\begin{verbatim}
procedure BinarySearch( a[], x ):
    // a[] - array from 0 to n-1
    low = 0
    high = n-1
    while low <= high:
        mid = low + (high - low) / 2
        if a[mid] < x:
            low = mid + 1
        else if a[mid] > x:
            high = mid - 1
        else
            return mid
    end
    return -1
end
\end{verbatim}
Пусть $n \le 2^k$ ($k = [\log n]$). Итераций: $k + 1$. На каждой итерации: $O(1)$. Итого:
$O(\log n)$.\\
{\bf Сортировка выбором (Selection sort):}\\
Шаги алгоритма:
\begin{enumerate}
\item Находим номер минимального (или максимального, смотря что хотим) значения в текущем списке
\item Производим обмен этого значения со значением первой неотсортированной позиции (обмен не нужен, если минимальный элемент уже находится на данной позиции)
\item Теперь сортируем хвост списка, исключив из рассмотрения уже отсортированные элемент
\end{enumerate}
Код:
\begin{verbatim}
procedure SelectionSort( a[] ):
    // a[] - array from 0 to n-1
    for i in [1..n-1]:
        j = index of max element in a[0..n-i]
    swap (a[j], a[n-i])
    end
end
\end{verbatim}
Количество сравнений:
$$
(n-1)+(n-2)+...+1 = \frac{n(n-1)}{2}
$$
Итого: $O(n^2)$.\\
{\bf Сортировка вставками (Insertion sort):}\\
Идея: для каждого элемента ищем место в отсортированной части массива, в которое его нужно вставить.\\
Код:
\begin{verbatim}
procedure InsertionSort( a[] ):
    for i in [1..n-1]:
        j = i, t = a[j]
        while j > 0 && t < a[j-1]:
            a[j] = a[j-1]
            j = j-1
        end
        a[j] = t
    end
end
\end{verbatim}
В лучшем случае сравнений
$$
1 + 1 + . . . + 1 = n - 1
$$
в худшем случае
$$
1 + 2 + . . . + (n - 1) = \frac{n(n - 1)}{2}
$$

\section {Сортировка слиянием. Основная теорема для метода «разделяй и властвуй»}

\section {Алгоритм Карацубы умножения чисел, алгоритм Штрассена умножения матриц. Быстрое преобразование Фурье}

\noindent{\bf Алгоритм Карацубы} \\
Пусть есть два n-значных двоичных числа $A = a x + b$ и $B=c x + d$, где n — чётное число и $x=10^{\frac n 2}$. То есть, a и c получены из старших $\frac n 2$ разрядов A и B, а b и d — из младших. \\
\noindent В таких обозначениях произведение чисел A и B может быть переписано как
$$
AB = (a x + b)(c x + d) = a c x^2 + (a d + b c) x + b d
$$

\noindent Таким образом, умножение n-значных чисел было сведено к четырём задачам умножения $\frac n 2$-значных чисел и сложению результатов, которые выполняются за $O(n)$.
Далее можем заметить, что на самом деле достаточно лишь трёх умножений $\frac n 2$-значных чисел, так как
$$
(a d + b c) = (a + b)(c + d) - a c - b d
$$
Таким образом, всё произведение AB может быть получено из $a c$, $b d$ и $(a + b)(c + d)$ линейными операциями сложения, вычитания, а общее время работы оценивается как
$$
T(n) = 3 T (\frac {n}{2}) + O(n) =  O(n^{\log _{2}3})
$$

\noindent{\bf Алгоритм Штрассена} 

\noindent Хотим перемножить две матрицы. Основная идея: сводим умножение двух матриц размера $n$ к семи умножениям матриц размерности $\frac n 2$
\begin{equation*}
XY = \left(
\begin{array}{cc}
A & B \\
C & D\\
\end{array}
\right)
\left(
\begin{array}{cc}
E & F \\
G & H \\
\end{array}
\right)
=
\left(
\begin{array}{cc}
AE + BG & AF + BH \\
CE + DG & CF + DH \\
\end{array}
\right)
\end{equation*}

\noindent Можно заметить, что 
\begin{equation*}
XY = \left(
\begin{array}{cc}
P_5 + P_4 - P_2 + P_6 & P_1 + P_2 \\
P_3 + P_4 & P_1+ P_5 - P_3 - P_7\\
\end{array}
\right)
\end{equation*}

\begin{equation*}
\begin{array}{llllll}
P_1 &=& A(F - H) & P_2 &=& (A + B)H \\
P_3 &=& (C + D)E & P_4 &=& D(G - E) \\
P_5 &=& (A + D)(E + H) & P_6 &=& (B - D)(G + H) \\
P_7 &=& (A - C)(E + F) &  \\
\end{array}
\end{equation*}

\noindent В терминах основной теоремы из второго билета имеем (т.к. сложение матриц - это $n^2$):
$$
T(n) = 7T(\frac{n}{2}) + O(n^2)
$$
\noindent Поэтому сложность равна $O(n^{\log_2 7})$

\noindent{\bf Быстрое преобразование Фурье} \\
Теперь хотим перемножать многочлены. Пусть даны $A(x) = a_0 + a_1 x + .. + a_n x^n$ и $B(x) = b_0 + b_1 x + .. + b_n x^n$. Их произведение обозначим за $C(x).$ \\

\noindent Заметим, что если многочлены заданы не в виде набора коэффициентов, а в виде значений в (2n + 1) различных точках, то умножение многочленов работает за линейное время (2n + 1, так как многочлен-произведение имеет степень 2n). Действительно, пусть $A(p_i) = \alpha_i$, $B(p_i) = \beta_i$, тогда $C(p_i) = A(p_i) B(p_i) = \alpha_i * \beta_i$ \\

\noindent Теперь если мы научимся переходить от коэффициентов многочлена к значениям и, наоборот, по значениям вычислять коэффициенты за $O(n \log n)$, то итоговое умножение тоже будет работать за $O(n \log n)$ \\

\noindent Заметим, что
\begin{equation*}
\begin{array}{lll}
A(x) &=& A_0(x^2) + x A_1(x^2)\text{, где} \\
A_0(X) &=& a_0 + a_2 X + a_4 X^2 + ... \\
A_1(X) &=& a_1 + a_3 X + a_5 X^2 + ...
\end{array}
\end{equation*} \\

\noindent Так как в представлении A мы считаем $A_0$ и $A_1$ от $x^2$, то выбрав симметричные относительно нуля точки мы ускоримся в подсчете значений по коэффициентам в два раза (т.к при подсчете $A(x)$ и $A(-x)$ вычисляются одни и те же $A_0(x^2)$ и $A_1(x^2)$) \\

\noindent Но для применения основной теоремы из второго билета это сработает только один раз, и перейти от $\frac n 2$ к $\frac n 4$ уже не получится, поскольку все точки теперь положительные. Проблема решается выходом в комплексную плоскость и вычислением значений многочлена A(x) в корнях из единицы степени n (там каждый раз половина точек будет склеиваться и к концу как раз останется одна точка - единица) \\

\noindent Теперь рассмотрим переход от значений в корнях из единицы к коэффициентам многочлена (обратное преобразование Фурье). Давайте перепишем то, что мы сделали (переход от коэффициентов к значениям, оно же прямое преобразование Фурье) в матричном виде:

\begin{equation*}
\left(
\begin{array}{c}
A(1) \\
A(\omega) \\
A(\omega^2) \\
\vdots \\
A(\omega^n) 
\end{array}
\right)
= 
\left(
\begin{array}{cccc}
1 & 1 & \ldots & 1 \\
1 & \omega & \ldots & \omega^n \\
1 & \omega^2 & \ldots & \omega^{2n}\\
\vdots & \vdots & \ddots & \vdots\\
1 & \omega^{n} & \ldots & \omega^{n^2} \\
\end{array}
\right)
\left(
\begin{array}{c}
a_0 \\
a_1 \\
a_2 \\
\vdots \\
a_n 
\end{array}
\right)
= M_n (\omega)
\left(
\begin{array}{c}
a_0 \\
a_1 \\
a_2 \\
\vdots \\
a_n 
\end{array}
\right)
\end{equation*}
Чтобы перейти от значений к коэффициентам нужно умножить слева это на $M_n^{-1} (\omega)$. Непосредственным умножением проверяется, что $M_n^{-1} (\omega) = \frac{1}{n} M_n (-\omega)$. То есть в матричном виде принципиально ничего не поменялось (появилось умножение на $\frac{1}{n}$ и $-w$ вместо $w$, но это никак принципиально не влияет на тот алгоритм, который был показан). Поэтому этот же алгоритм применим и для получения коэффициентов по значениям.

\noindentВ терминах основной теоремы имеем (для преобразований Фурье):
$$
T(n) = 2T(\frac n 2) + O(n)
$$
То есть общая сложность $O(n \log n)$ \\

\noindent {\bf P.S.} Это все работает, когда n - степень двойки. Если это не так, то просто дополним старшие коэффициенты многочлена нулями до ближайшей степени двойки. Очевидно, что степень многочлена увеличится не более чем в два раза, что не влияет на ассимптотику, т.к $O(2 n \log(2 n)) = O(n \log n)$

\section {Оценка снизу количества сравнений при сортировке, быстрая сортировка Хоара, сложность в среднем и в худшем. Задача Дейкстры о голландском флаге, 3х-частное разбиение, эвристики выбора опорного элемента} 

\noindent {\bf Теорема.} Любой детерменированный алгоритм сортировки сравнением имеет в худшем случае $\Omega (n log n)$ \\

\noindent Количество перестановок в массиве из n элементов $= n!$ \\
Работу алгоритма на различных входных данных можно
представить в виде бинарного дерева. Каждое ветвление –
сравнение элементов массива. Если при любых входных данных
количество сравнений не больше S, то глубина дерева не больше
S.\\

\noindent Оценим теперь глубину этого дерева. По определению глубины дерева, в дереве глубины S кол-во путей $\le 2^S$. C другой стороны, так как сортировка должна работать для любого массива, то итоговое кол-во путей должно быть $\ge n!$. Поэтому $2^S \ge n!$, то есть $s \ge log_2 n! \ge log_2 \frac{n}{2}^{\frac{n}{2}} \ge k n log n$ для некоторого k. \\

\noindent{\bf Быстрая сортировка:}

\noindent Алгоритм состоит из трёх шагов:
\begin{enumerate}
    \item Выбрать элемент из массива. Назовём его опорным
    \item Разбиение: перераспределение элементов в массиве таким образом, что элементы меньше опорного помещаются перед ним, а больше или равные после
    \item Рекурсивно применить первые два шага к двум подмассивам слева и справа от опорного элемента. Рекурсия не применяется к массиву, в котором только один элемент или отсутствуют элементы
\end{enumerate}

\begin{verbatim}
procedure QuickSort(a[], left, right):
    // a[] - массив от 0 до n-1
    pivot = a[left]
    i = left + 1
    for j in [left + 1 .. right]:
        if a[j] < pivot:
            переставить a[j] и a[i]
            i++
        end
    end
    переставить a[left] и a[i-1]
    
    QuickSort(a, left, i - 2)
    QuickSort(a, i, right)
end
\end{verbatim} \\

\noindent Сложность
\begin{itemize}
    \item В худшем случае: $O(n^2)$
    \item В лучшем случае (опорный элемент - медиана):
    $$
    T(n) = 2T(\frac{n}{2})+ O(n) \Rightarrow T(n) = O(n log n)
    $$
    \item В «плохом» случае - опорный элемент делит массив в соотношении 99/100: $O(n log n)$
\end{itemize}

\noindent{\bf Теорема.} Сложность быстрой сортировки «в среднем» (при случайном выборе опорного элемента) равно $O (n log n)$

\noindent Док-во: есть в презентации, так что скатать можно оттуда, но если будет время - перенесу \\

\noindent Свойства:
\begin{itemize}
    \item Требует $\approx log n$ дополнительной памяти
    \item Не является устойчивым (может не сохранять порядок одинаковых элементов)
\end{itemize}

\noindent{\bf Трехчастное разбиение:} 

\noindent Заметим, что сложность быстрой сортировки на массиве из одинаковых элементов $= O(n^2)$. Решение - трехчастное разбиение:
\begin{itemize}
    \item a[i] < p для $0 \le i\le r$ - меньше опорного
    \item a[i] = p для $r + 1 \le i \le s$ - равны опорному
    \item a[i] > p для $s + 1 \le i \le n - 1$ - больше опорного
\end{itemize}

\begin{verbatim}
procedure QuickSort3Way(a[]):
    // a[] - массив от 0 до n-1
    перемешать (shuffle) массив a
    QuickSort3Way(a, 0, n-1)
end

procedure QuickSort3Way(a[], low, high):
    p = a[low]
    i = low + 1
    lt = low + 1
    gt = high
    while i <= gt:
        if a[i] < p:
            Exch(a, lt++, i++)
        else if a[i] > p:
            Exch(a, i, gt--)
        else
            i++
    end
    Exch(a, low, --lt)
    QuickSort3Way(a, low, lt - 1)
    QuickSort3Way(a, gt + 1, high)
end
\end{verbatim}

\noindent{\bf Задача о голландском флаге:} отсортировать массив, состоящий только из нулей, единиц и двоек. Как раз для этой задачи трехчастное разбиение идеально подходит.

\noindent{\bf Практические улучшения aka эвристики:}

\begin{enumerate}
    \item На массивах небольшого размера выгоднее использовать сортировку вставками
    \item Выбор опорного элемента: медиана из трех (median-of-3). Выбираем 3 случайны элемента и в качестве опорного берем второй по порядку (медиану).
    \item Выбор опорного элемента: выбираем 3 тройки случайных элементов, в
    каждой тройке находим медиану и в качестве опорного берем медиану медиан.
\end{enumerate}

\section {Алгоритмы нахождения $k$-й порядковой статистики - вероятностный и детерминированный.
Метод сортировки TimSort.}

\noindent\textbf{Определение}\\
k-ой порядковой статистикой набора элементов линейно упорядоченного множества называется такой его элемент, который является $k$-ым элементом набора в порядке сортировки.

\subsection{Модфифкация Quicksort (basic)}
\noindent\textbf{Описание алгоритма}:\\
Будем использовать процедуру рассечения массива элементов из алгоритма сортировки QuickSort. Пусть нам надо найти $k$-ую порядковую статистику, а после рассечения опорный элемент встал на позицию m. Возможно три случая:
\begin{itemize}
    \item $k = m$: Порядковая статистика найдена.\\
    \item $k < m$: Рекурсивно ищем $k$ -ую статистику в первой части массива.\\
	\item $k > m$: Рекурсивно ищем $(k - m -1)$ -ую статистику во второй части массива.\\
\end{itemize}
\noindent\textbf{Анализ времени работы}: $O(n)$
\subsection{Поиск k-ой порядковой статистики за линейное время(BFPRT)}

\noindent\textbf{Идея}:\\
Этот алгоритм является модификацией алгоритма поиска $k$-ой порядковой статистики. Важное отличие заключается в том, что время работы алгоритма в наихудшем случае — $O(n)$, где $n$ — количество элементов в множестве. Главная идея алгоритма заключается в том, чтобы гарантировать хорошее разбиение массива. Алгоритм выбирает такой рассекающий элемент, что количество чисел, которые меньше рассекающего элемента, не менее $\frac{3n}{10}$. Элементов же больших опорного элемента, также не менее $\frac{3n}{10}$. Благодаря этому алгоритм работает за линейное время в любом случае. 

\noindent\textbf{Описание алгоритма}:\\
\begin{itemize}
    \item[1] Все $n$ элементов входного массива разбиваются на группы по пять элементов, в последней группе будет $n mod 5$ элементов. Эта группа может оказаться пустой при $n$ кратным 5.
    \item[2] Сначала сортируется каждая группа, затем из каждой группы выбирается медиана.
	\item[3] Путем рекурсивного вызова шага определяется медиана $x$ из множества медиан (верхняя медиана в случае чётного количества), найденных на втором шаге. Найденный элемент массива $x$ используется как рассекающий (за $i$ обозначим его индекс).
	\item[4] Массив делится относительно рассекающего элемента x.
	\item[5] Если $i = k$, то возвращается значение $x$. Иначе запускается рекурсивно поиск элемента в одной из частей массива: $k$-ой статистики в левой части при $i > k$ или $(k - i - 1)$-ой статистики в правой части при $i < k$.
\end{itemize}

\noindent\textit{Источник}: \href{http://neerc.ifmo.ru/wiki}{Тыкай Алгоритмы и стркутуры данных-> алгоритмы поиска}

\subsection{Timsort}
\noindent\textbf{Идея}:\\
Алгоритм Timsort состоит из нескольких частей: 
\begin{itemize}
    \item[1] Входной массив разделяется на подмассивы фиксированной длины, вычисляемой определённым образом.
    \item[2]  Каждый подмассив сортируется сортировкой вставками, сортировкой пузырьком или любой другой устойчивой сортировкой.
	\item[3]  Отсортированные подмассивы объединяются в один массив с помощью модифицированной сортировки слиянием.
\end{itemize}


\noindent\textbf{Обозначения}:\\
Алгоритм Timsort состоит из нескольких частей: 
\begin{itemize}
    \item[1] $n$ — размер входного массива.
    \item[2]  $run$ — подмассив во входном массиве, который обязан быть упорядоченным одним из двух способов: строго по убыванию или нестрого по возрастанию
	\item[3]  $minrun$ — минимальный размер подмассива, описанного в предыдущем пункте.
\end{itemize}


\noindent\textbf{Общая схема}:\\

\begin{itemize}
    \item[1] Определяем $minrun$, текущая позиция $= 1$
    \item[2] Находим $run$, начинающийся с текущей позиции
	\item[3] Если $run$ убывающий \--- переворачиваем
	\item[4] Если $run$ короче $minrun$ \--- дополняем и сортируем вставками
	\item[5] Добавляем в стек (начало подмассива, длина)
	\item[6] Проверяем условия баланса стека 
	\item[7] Если не дошли до конца массива \--- переходим к 2
	\item[8] Сливаем подмассивы, хранящиеся в стеке
\end{itemize}

\noindent\textbf{Слияние}:\\

\begin{itemize}
    \item[1]  Создается пустой стек пар < индекс начала подмассива, размер подмассива >.
    \item[2] Берется первый упорядоченный подмассив.
	\item[3] Добавляется в стек пара данных < индекс начала текущего подмассива, его размер >.
	\item[4]  Пусть $X$,$Y$,$Z$ — длины верхних трех интервалов, которые лежат в стеке. Причем $X$ — это последний элемент стека (если интервалов меньше трёх, проверяем лишь условия с оставшимися интервалами).
	\item[5] Повторяем пока выражение $(Z>X+Y \wedge Y>X)$ не станет истинным: если размер стека не меньше 2 и $Y \leq X$ — сливаем $X$ c $Y$, а если размер стека не меньше 3 и $Z \leq X+Y$ — сливаем $Y$ c $min(X,Z)$. 
	\item[6]  Переходим к шагу 3 
\end{itemize}
	
	\noindent\textbf{Описание процедуры слияния}:\\

\begin{itemize}
    \item[1]  Создается временный массив в размере меньшего из сливаемых подмассивов.
    \item[2]  Меньший из подмассивов копируется во временный массив, но надо учесть, что если меньший подмассив — правый, то ответ (результат сливания) формируется справа налево. Дабы избежать данной путаницы, лучше копировать всегда левый подмассив во временный. На скорости это практически не отразится. 
	\item[3] Ставятся указатели текущей позиции на первые элементы большего и временного массива.
	\item[4]  На каждом шаге рассматривается значение текущих элементов в большем и временном массивах, берется меньший из них, копируется в новый отсортированный массив. Указатель текущего элемента перемещается в массиве, из которого был взят элемент.
	\item[5] Предыдущий пункт повторяется, пока один из массивов не закончится.
	\item[6] Все элементы оставшегося массива добавляются в конец нового массива.
\end{itemize}

\noindent\textit{Есть модификация слияния - Galloping mode}

\noindent\textbf{Время}: $O(n log n)$
\section {Абстрактные типы данных, интерфейс и реализация, стек и очередь - реализация связанным списком и массивом.}

\noindent\textbf{АТД} \--- тип данных, представление которого скрыто от клиента (= интерфейс).

При использовании \--- особое внимание уделяется операциям, указанным там. При реализации \--- данным, а потом уже реализуются операции над ними. \\

\noindent\textbf{Стек}:
\begin{itemize}
\item LIFO
\item pop, push, peek(доступ к верхнему элементу) и count
\item нет итератора


\end{itemize}

\noindent\textbf{Очередь}:
\begin{itemize}
\item FIFO
\item enqueue, dequeue, peek(доступ к верхнему элементу) и count
\item тоже нет итератора
\end{itemize}

\noindent\textbf{Двусторонняя очередь (Дэк)}:
\begin{itemize}
\item удаляем и добавляем с двух сторон
\item enqueue, dequeue, peek(доступ к верхнему элементу) - для начала и для конца и count
\item тоже нет итератора
\end{itemize}


\noindent\textbf{Реализация стека на основе связанного списка}:
\begin{verbatim}
public class Stack
{
    LinkedList _items = new LinkedList();

    public void Push(T value)
    {
        _items.AddLast(value);
    }

    public T Pop()
    {
        if (_items.Count == 0)
    	{
        	throw new InvalidOperationException("The stack is empty");
    	}

    	T result = _items.Tail.Value;

    	_items.RemoveLast();

    	return result;
    }

    public T Peek()
    {
        if (_items.Count == 0)
    	{
        	throw new InvalidOperationException("The stack is empty");
    	}

    	return _items.Tail.Value;
    }

    public int Count
    {
        get
   		{
        	return _items.Count;
    	}
    }
}
\end{verbatim}
\noindent\textit{Для очереди примерно то же самое}

\noindent\textbf{Реализация дека на основе массива}:
\begin{verbatim}
public class Stack
{
    T[] _items = new T[0];

    // Количество элементов в очереди.
    int _size = 0;

    // Индекс первого (самого старого) элемента.
    int _head = 0;

    // Индекс последнего (самого нового) элемента.
    int _tail = -1;
    }

	//интерфейс такой же, писать не надо
}
\end{verbatim}

\noindent\textit{Source}: \href{https://tproger.ru/translations/stacks-and-queues-for-beginners/amp/}{примеры кода и все остальное}
\section {Ассоциативные массивы, бинарные деревья поиска.}
\noindent{\bf Ассоциативный массив (Map, Dictionary)}\\
Абстрактный тип данных:

\qquad набор пар <ключ, значение> (ключ уникальный).

Операции (интерфейс):
\begin{enumerate}
\item Вставка
\item Поиск по ключу
\item Удаление по ключу
\end{enumerate}

\begin{verbatim}
interface Map<K, V> {
    ...
    V get(K key);
    V put(K key, V value);
    V remove(K key);
    ...
}
\end{verbatim}

{\bf Реализация ассоциативного массива}
\begin{enumerate}
\item Массив
\item Отсортированный массив
\item Бинарные деревья поиска
\item Хеш-таблицы
\end{enumerate}
{\bf C\#: интерфейс IDictionary}
\begin{verbatim}
public interface IDictionary<TKey, TValue> :
    ICollection<KeyValuePair<TKey, TValue>>,
    IEnumerable<KeyValuePair<TKey, TValue>>
{
    ICollection<TKey> Keys { get; }
    ICollection<TValue> Values { get; }
    
    TValue this[TKey key] { get; set; }
    void Add(TKey key, TValue value);
    bool ContainsKey(TKey key);
    bool Remove(TKey key);
    bool TryGetValue(TKey key, out TValue value);
}

public interface ICollection<T> : IEnumerable<T>, IEnumerable
{
    int Count { get; }
    void Clear();
    ...
}
\end{verbatim}
{\bf Java: интерфейс Map}
\begin{verbatim}
public interface Map<K,V> {
    int size();
    boolean isEmpty();
    boolean containsKey(Object key);
    boolean containsValue(Object value);
    V get(Object key);
    V put(K key, V value);
    V remove(Object key);
    void clear();

    Set<K> keySet();
    Collection<V> values();
    Set<Map.Entry<K, V>> entrySet();
    
    interface Entry<K,V> {
        K getKey();
        V getValue();
        V setValue(V value);
    }
}
\end{verbatim}
{\bf Java: интерфейс Set}
\begin{verbatim}
public interface Set<E> implements Iterable<E> {
    int size();
    boolean isEmpty();
    boolean contains(Object o);
    V get(Object key);
    boolean add(E e);
    V remove(Object o);
    void clear();
    ...
}
\end{verbatim}
{\bf Стандартные классы в Java и C\#}\\
\underline{Java:} \\
HashSet<T>, HashMap<K, V> \\
TreeSet<T>, TreeMap<K, V>

\begin{verbatim}
public abstract class AbstractSet<E> implements Set<E>
public class HashSet<E> extends AbstractSet<E>
public class TreeSet<E> extends AbstractSet<E>

public abstract class AbstractMap<K,V> implements Map<K,V>
public class HashMap<K,V> extends AbstractMap<K,V>
public class TreeMap<K,V> extends AbstractMap<K,V>
\end{verbatim}
\underline{C\#:} \\
HashSet<T>, Dictionary<K,V> \\
SortedSet<T>, SortedDictionary<K,V>\\
{\bf Подсчет количества слов}
\begin{verbatim}
private static void CountWords()
{
    var dic = new Dictionary<string, int>();
    while (true)
    {
        string line = Console.ReadLine();
        if (string.IsNullOrEmpty(line))
        break;
        
        string[] words = line.Split(’ ’);
        foreach (string word in words)
        {
            if (dic.ContainsKey(word))
                dic[word] = dic[word] + 1;
            else
                dic[word] = 1;
        }
    }
    
    foreach (var kvp in dic)
        Console.WriteLine("{0} = {1}", kvp.Key, kvp.Value);
}
\end{verbatim}

\noindent{\bf Бинарное дерево поиска}\\
\textit{Двоичное дерево поиска} (англ. binary search tree, BST) — это двоичное дерево, для которого выполняются следующие дополнительные условия (свойства дерева поиска):

\begin{enumerate}
\item Оба поддерева — левое и правое — являются двоичными деревьями поиска.
\item У всех узлов \textit{левого} поддерева произвольного узла X значения ключей данных \textit{меньше либо равны}, нежели значение ключа данных самого узла X.
\item У всех узлов \textit{правого} поддерева произвольного узла X значения ключей данных \textit{больше либо равны}, нежели значение ключа данных самого узла X.
\end{enumerate}
(Ключи уникальны, поэтому отношение строго <, наверное)\\
Вершины - записи вида (data, left, right), иногда еще нужен parent, data - пара (key, value)\\
Базовый интерфейс двоичного дерева поиска состоит из трёх операций (сложность из википедии):
\begin{enumerate}
\item FIND(K) — поиск узла, в котором хранится пара (key, value) с key = K. (Сложность в среднем  O(log n), в худшем случае~-- O(n))
\item INSERT(K, V) — добавление в дерево пары (key, value) = (K, V). (В ср.~-- O(log n), в худ~-- O(n))
\item REMOVE(K) — удаление узла, в котором хранится пара (key, value) с key = K. (В ср.~-- O(log n), в худ~-- O(n))
\end{enumerate}
{\bf Поиск элемента (FIND)}\\
\textit{Дано:} дерево Т и ключ K.\\
\textit{Задача:} проверить, есть ли узел с ключом K в дереве Т, и если да, то вернуть ссылку на этот узел.\\
\textit{Алгоритм:}
\begin{itemize}
\item Если дерево пусто, сообщить, что узел не найден, и остановиться.
\item Иначе сравнить K со значением ключа корневого узла X.
\begin{itemize}
\item[\labelitemi] Если K=X, выдать ссылку на этот узел и остановиться.
\item[\labelitemi] Если K>X, рекурсивно искать ключ K в правом поддереве Т.
\item[\labelitemi] Если K<X, рекурсивно искать ключ K в левом поддереве Т.
\end{itemize}
\end{itemize}
{ \bf Добавление элемента (INSERT)}\\
\textit{Дано:} дерево Т и пара (K, V).\\
\textit{Задача:} вставить пару (K, V) в дерево Т (при совпадении K, заменить V).\\
\textit{Алгоритм:}
\begin{itemize}
\item Если дерево пусто, заменить его на дерево с одним корневым узлом ((K, V), null, null) и остановиться.
\item Иначе сравнить K с ключом корневого узла X.
\begin{itemize}
\item[\labelitemi] Если K>X, рекурсивно добавить (K, V) в правое поддерево Т.
\item[\labelitemi] Если K<X, рекурсивно добавить (K, V) в левое поддерево Т.
\item[\labelitemi] Если K=X, заменить V текущего узла новым значением.
\end{itemize}
\end{itemize}
{ \bf Удаление узла (REMOVE)}\\
\textit{Дано:} дерево Т с корнем n и ключом K.\\
\textit{Задача:} удалить из дерева Т узел с ключом K (если такой есть).\\
\textit{Алгоритм:}\\
 \noindent 1. Если дерево T пусто, остановиться;\\
2. Иначе сравнить K с ключом X корневого узла n.
\begin{itemize}
\item[\labelitemi] Если K>X, рекурсивно удалить K из правого поддерева Т;
\item[\labelitemi] Если K<X, рекурсивно удалить K из левого поддерева Т;
\item[\labelitemi] Если K=X, то необходимо рассмотреть три случая.
\begin{itemize}
\item[\labelitemi] Если обоих детей нет, то удаляем текущий узел и обнуляем ссылку на него у родительского узла;
\item[\labelitemi] Если одного из детей нет, то значения полей ребёнка m ставим вместо соответствующих значений корневого узла, затирая его старые значения, и освобождаем память, занимаемую узлом m;
\item[\labelitemi] Если оба ребёнка присутствуют, то
\begin{itemize}
\item[\labelitemi] Если самый левый элемент правого поддерева m не имеет поддеревьев
\begin{itemize}
\item[\labelitemi] Копируем значения K, V из m в удаляемый элемент
\item[\labelitemi] Удаляем m
\end{itemize}
\item[\labelitemi] Если m имеет правое поддерево
\begin{itemize}
\item[\labelitemi] Копируем значения K, V из m в удаляемый элемент
\item[\labelitemi] Заменяем у родительского узла ссылку на m ссылкой на правое поддерево m
\item[\labelitemi] Удаляем m
\end{itemize}
\end{itemize}
\end{itemize}
\end{itemize}
{\bf Реализация ("Algorithms in Java")}
\begin{verbatim}
public class BST<Key extends Comparable<Key>, Value>
{
    private class Node
    {
        private Key key; // key
        private Value val; // associated value
        private Node left, right; // links to subtrees
        private int N; // # nodes in subtree rooted here
        
        public Node(Key key, Value val, int N) {
            this.key = key; this.val = val; this.N = N;
        }
    }
    
    private Node root; // root of BST
    
    public int size() { return size(root); }
    private int size(Node x) { return x == null ? 0 : x.N; }
    
    public Value get(Key key) { ... }
    public void put(Key key, Value val) { ... }
}
\end{verbatim}
Бинарные деревья поиска используют сравнения
\begin{verbatim}
interface Comparable<T> {
int compareTo(T o);
}
public interface Comparator<T> {
int compare(T o1, T o2);
}
\end{verbatim}
\noindent Java: Comparable, Comparator \\
C\#: IComparable, IComparer\\
{\bf Реализация: get}
\begin{verbatim}
public Value get(Key key) {
    return get(root, key);
}

private Value get(Node x, Key key) {
    // Return value associated with key in the subtree rooted at x;
    // return null if key not present in subtree rooted at x.
    if (x == null) return null;
    
    int cmp = key.compareTo(x.key);
    if (cmp < 0) return get(x.left, key);
    else if (cmp > 0) return get(x.right, key);
    else return x.val;
}
\end{verbatim}
{\bf Реализация: put}
\begin{verbatim}
public void put(Key key, Value val) {
    // Search for key. Update value if found; grow table if new.
    root = put(root, key, val);
}

private Node put(Node x, Key key, Value val) {
    // Change key’s value to val if key in subtree rooted at x.
    // Otherwise, add new node to subtree associating key with val.
    if (x == null) return new Node(key, val, 1);
    
    int cmp = key.compareTo(x.key);
    if (cmp < 0) x.left = put(x.left, key, val);
    else if (cmp > 0) x.right = put(x.right, key, val);
    else x.val = val;
    
    x.N = size(x.left) + size(x.right) + 1;
    return x;
}
\end{verbatim}
(Тут какая-то дичь с симуляциями, думаю, что это лучше прямо в презенташке смотреть)\\
{\bf Сложность в случайном BST}\\
\textit{Утв.} Search hit в случайном бинарном дереве поиска,
содержащем N ключей требует в среднем $\sim 2 ln N (\approx 1.39 log N)$ сравнений.\\
%\vspace{3mm}
Пусть $C_N$ – сумма длин путей до всех узлов в дереве с $N$ узлами (тогда успешный поиск требует в среднем $1 + C_N /N$).
\begin{center}
$C_N = N-1+(C_0+C_{N-1})/N+(C_1+C_{N-2})/N+. . .+(C_{N-1}+C_0)/N$\\
$C_0 = C_1 = 0$
\end{center}
(рекурентное соотношение аналогично тому, которое возникало при анализе Quicksort, см. лекцию 3).
\vspace{2mm}
\begin{center}
$C_N \sim 2N ln N$
\end{center}
\vspace{3mm}
\textit{Утв.} Вставка и search miss в случайном бинарном дереве
поиска, содержащем $N$ ключей требует в среднем $\sim 2 ln N (\approx 1.39 log N)$ сравнений.\\
Вставка и search miss требует на 1 сравнение больше, чем
search hit (задача).

\vspace{3mm}
\noindent{\bf Hibbard deletion (Удаление элемента)}\\
Один из эффективных способов сделать это, обнаруженный Томасом Хиббардом в 1962 году, состоит в том, чтобы заменить удаляемый узел местами его преемника . Преемником ключа является следующий по величине ключ в дереве, и его удобно найти как минимальный ключ в правом поддереве удаляемого узла.

Так же в презенташке написано Tombstone в удалении, но я толком не нашла ничего.

\section{Сбалансированные, 2-3 и красно-черные деревья}

\subsection*{Сбалансированные деревья}
Идеально сбалансированное дерево = все пути от корня до
конечных вершин имеют одинаковую длину.
В реальности так не бывает почти никогда(!), поэтому хочется, чтобы min/max длины отличались не сильно.

Два способа этого достичь:

\begin{itemize}
\item AVL (Классическое определение сбалансированного дерева)
Для любого узла разница между высотами левого и правого
поддерева не превосходит 1.
\item Красно-черные Количество черных ребер в любом пути одинаково,
количество красных ребер в пути не превосходит количество
черных + 1.
\end{itemize}

Отсюда гарантированная сложность - $O(logN)$

\subsection*{2-3 деревья}

{\bf Структура:} каждый узел может содержать 1 ключ и $\le 2$ потомков или 2 ключа и $\le 3$ потомков. Ключи отсортированны внутри узла. В 2-узле левое поддерево содержит ключи меньше ключа в узле, правое - больше. В 3-узле левое поддерево содержит ключи меньшие левого ключа узла, правое - большие правого ключа, среднее - большие левого, но меньшие правого.

{\bf Вставка:} 

1. Выполняем поиск, находим лист, в который надо
вставить ключ.

2. Если лист является 2-узлом, то превращаем его в 3-узел.

3. Если лист является 3-узлом, то превращаем в 4-узел.

4. 4-узел хранит 3 ключа – средний ключ перемещаем в
родителя, левый и правый ключи превращаем в 2-узлы.

5. Если родитель был 3-узлом, то он превратился в 4-узел
– выполняем для него шаг 4.
 
{\bf Удаление:} (Не помню, чтобы было на лекциях, однако алгоритм такой же как в B-дереве в билете 9, так как 2-3 дерево - частный случай B-дерева с M = 3).

\subsection*{Красно-черные деревья}
Идея: в 2-3 дереве заменить 3-узлы на 2 узлы с дополнительным "красным" ребром (идущим, например, влево для левостороннего к-ч дерева).

{\bf Свойства:}
\begin{itemize}
\item Количество черных ребер в любом пути одинаково
\item Все красные ребра идут налево
\item Нет двух последовательных красных ребер
\item Взаимнооднозначно соответствуют 2-3 деревьям
\item Реализация get – как в обычном BST
\end{itemize}

{\bf Вставка:}

Алгоритм будет показан как программная реализация:

\begin{verbatim}
private static final boolean RED = true;
private static final boolean BLACK = false;

private class Node
{
    Key key;
    Value val;
    Node left, right;
    boolean color; // color of parent link
}

private boolean isRed(Node x)
{
    if (x == null) return false;
    return x.color == RED;
}
\end{verbatim}

3 вспомогательных операции:


\begin{verbatim}
private Node rotateLeft(Node h)
{
    // assert isRed(h.right);
    Node x = h.right;
    h.right = x.left;
    x.left = h;
    x.color = h.color;
    h.color = RED;
    return x;
}
\end{verbatim}


\begin{verbatim}
private Node rotateRight(Node h)
{
    // assert isRed(h.left);
    Node x = h.left;
    h.left = x.right;
    x.right = h;
    x.color = h.color;
    h.color = RED;
    return x;
}

\end{verbatim}


\begin{verbatim}
private void flipColors(Node h)
{
    // assert !isRed(h);
    // assert isRed(h.left);
    // assert isRed(h.right);
    h.color = RED;
    h.left.color = BLACK;
    h.right.color = BLACK;
}
\end{verbatim}

Задача - сохранить свойства к-ч дерева при вставке, из этого вытекает следующая функция вставки:

\begin{verbatim}
private Node put(Node h, Key key, Value val)
{
    if (h == null) return new Node(key, val, RED);
    
    int cmp = key.compareTo(h.key);
    if (cmp < 0) h.left = put(h.left, key, val);
    else if (cmp > 0) h.right = put(h.right, key, val);
    else if (cmp == 0) h.val = val;
    
    // Все красные ребра идут налево
    if (isRed(h.right) && !isRed(h.left)) h = rotateLeft(h);
    // Не может быть два красных ребра подряд     
    if (isRed(h.left) && isRed(h.left.left)) h = rotateRight(h);
    // Все красные ребра идут налево 
    if (isRed(h.left) && isRed(h.right)) flipColors(h);          
    
    return h;
}
\end{verbatim}

{\bf Удаление:} (На лекциях не рассказывалось)


\section{B-деревья.}

{\bf Применение}: часто бывают ситуации, в которых чтение узла дерева занимает относительно большое время, и при этом чтение выполняется поблочно, для этого хочется уменьшить число узлов и увеличить их размер.
(Файловые системы, базы данных)

{\bf Структура:} B-деревья обобщают 2-3 - деревья: в вершинах дерева хранится от $\frac{M}{2}$ до $M-1$ ключей, за исключением корневой вершины, в которой может храниться от $0$ до $M-1$ ключей. 
В каждой вершине - 2 массива: отсортированный массив ключей и массив данных (на 1 больший по размеру). "Внешние узлы хранят ключи (+данные), внутренние узлы хранят ключи для обеспечения поиска". 

Моя интерпретация:
В узлах, не являющихся листами, $i$-ое значение данных, является ссылкой на поддерево, содержащее ключи в интервале $(K_{i-1}, K_i)$, для $i = 0$ интервал - $(-\inf, K_0)$, для последнего $i = s+1$ интервал - $(K_s, +\inf)$. В узлах, являющихся листами, хранятся реальные данные. 

B-дерево - сбалансированное, откуда сложность поиска от $log_{M-1}{N}$ до $log_{M/2}{N}$. Опишем кратко операции поиска, добавления и удаления элемента.

\subsection*{Поиск}
Поиск ключа: находясь в узле проверяем, есть ли искомый ключ в отсортированном массиве ключей узла: если - да, то ключ найден, иначе  - определяем между какими ключами узла лежит искомый и спускаемся в соответствующее поддерево.

\subsection*{Добавление}
Вначале определим функцию, которая добавляет ключ K к поддереву потомков узла $x$.

Если $x$ — не лист,
	\begin{itemize}
        \item Определяем интервал, где должен находиться K. Пусть $y$ — соответствующий потомок.
        \item Рекурсивно добавляем K к дереву потомков $y$.
        \item Если узел $y$ полон, то есть содержит $M-1$ ключей, расщепляем его на два. Узел $y_1$ получает первые $\frac{M}{2}-1$ из ключей $y$ и первые $\frac{M}{2}$ его потомков, а узел $y_2$ — последние $\frac{M}{2}-1$ из ключей $y$ и последние $\frac{M}{2}$ его потомков. Средний по порядку из ключей узла $y$ попадает в узел $x$, а указатель на $y$ в узле $x$ заменяется указателями на узлы $y_1$ и $y_2$.
	\end{itemize}    	
	
	Если $x$ — лист, просто добавляем туда ключ K (при переполнении аналогично расщепляем и выталкиваем средний по порядку элемент вверх)

Теперь определим добавление ключа K ко всему дереву.

    Добавим K к дереву потомков корневого узла.

    Если Root содержит теперь $M-1$ ключей, расщепляем его на два. Узел $R_{1}$ получает первые $\frac{M}{2}-1$ из ключей Root и первые $\frac{M}{2}$ его потомков, а узел $R_{2}$ — последние $\frac{M}{2}-1$ из ключей Root и последние $\frac{M}{2}$ его потомков. Средний по порядку из ключей узла Root попадает вo вновь созданный узел, который становится корневым. Узлы $R_{1}$ и $R_{2}$ становятся его потомками.

\subsection*{Удаление}
(На лекциях, кажется, не было, но вот краткое описание:)

Если корень одновременно является листом, то есть в дереве всего один узел, мы просто удаляем ключ из этого узла. В противном случае сначала находим узел, содержащий ключ, запоминая путь к нему. Пусть этот узел — $x$.

Если $x$ — лист, удаляем оттуда ключ. Если в узле $x$ осталось не меньше $t-1$ ключей, мы на этом останавливаемся. Иначе мы смотрим на количество ключей в следующем узле (брате), а потом в предыдущем узле (брате). Если следующий узел есть, и в нём не менее $t$ ключей, мы добавляем в $x$ ключ-разделитель между ним и следующим узлом из родительского узла, а на его место ставим первый ключ следующего узла, после чего останавливаемся. Если это не так, но есть предыдущий узел, и в нём не менее $t$ ключей, мы добавляем в $x$ ключ-разделитель между ним и предыдущим узлом, а на его место ставим последний ключ предыдущего узла, после чего останавливаемся. Наконец, если и с предыдущим ключом не получилось, мы объединяем узел $x$ со следующим или предыдущим узлом, и в объединённый узел перемещаем ключ, разделяющий два узл в родительском узле. При этом в родительском узле может остаться только $t-2$ ключей. Тогда, если это не корень, мы выполняем аналогичную процедуру с ним. Если мы в результате дошли до корня, и в нём осталось от 1 до $t-1$  ключей, делать ничего не надо, потому что корень может иметь и меньше $t-1$ ключей. Если же в корне не осталось ни одного ключа, исключаем корневой узел, а его единственный потомок делаем новым корнем дерева.


Если $x$ — не лист, а K — его $i$-й ключ, удаляем самый правый ключ из поддерева потомков $i$-го потомка $x$, или, наоборот, самый левый ключ из поддерева потомков $i+1$-го потомка $x$. После этого заменяем ключ K удалённым ключом. Удаление ключа происходит так, как описано в предыдущем абзаце. 


\section{Хеш-таблицы, реализация методом цепочек и открытой адресацией.}

\subsection*{хеш-таблицы}

{\bf Ассоциативный массив} - структура данных, хранящая наборы типа <Key, Value>. 
Предполагается, что ассоциативный массив не может хранить две пары с одинаковыми ключами (на практике этого добиться сложно).
Интерпретация ассоциативного массива - отображение $value: Key \rightarrow Value$.

Операции (или интерфейс):
\begin{itemize}
\item Вставка (добавление)
\item Поиск по ключу (взятие значения или проверка)
\item Удаление по ключу
\end{itemize}

{\bf Хеш-таблица} - структура, реализующая ассоциативный массив.

{\bf Важные свойства} хеш-таблиц состоит в том, что, при некоторых разумных допущениях, в среднем:
\begin{itemize}
\item Вставка: O(1)
\item Поиск по ключу: O(1)
\item Удаление по ключу: O(1)
\item Память: O(n)
\end{itemize}

Отображение $h: U \rightarrow \{0, 1, ..., k\}$, где $U$ - множество ключей называется {\bf хеш-функций}.
Если $h$ - хеш-функция, то можно считать, что $value(u) = array[h(u)]$, где $array$ - массив размера $m$.

Пример:
$"abc" \rightarrow ord(a)*ord(b)*ord(c)$.

Код:
\begin{verbatim}
public int hash(char[] value) 
{
	int h = 0;
	for (int i = 0; i < value.length; i++) 
	{
		h = 31 * h + value[i];
	}
	return h;
}
\end{verbatim}

В таком определении могут возникнуть {\bf коллизии} $h(u_1) = h(u_2)$, где $u_1, u_2 \in U$.

\subsection*{Разрешение коллизий. Метод цыпочек}
Вычисляем $h(u)$. В случае коллизии пихаем элемент в "соседнюю" (это понятие резиновое, например правая или вторая справа. В общем случае это {\bf последовательность проб} $h_0(x), h_1(x), …, h_{n-1}(x)$) ячейку.
Как искать? - пробегаем все соседнии ячейки, пока ключ совпадает.

\subsubsection*{Теорема}
Математическое ожидание сложности неудачного поиска
при условии простого равномерного хеширования равна
$O(1 + \alpha)$, где $\alpha$ - коэффициент заполнения таблицы.
$\alpha = \frac{n}{m}$, где $m$ - число ячеек. 

\subsection*{Разрешение коллизий. Открытой адресации}
$array[h(u)]$ - список. В случае коллизий append-им пару (ключ, значение) туда.


\subsubsection*{Теорема}
Математическое ожидание сложности неудачного поиска
при условии простого равномерного хеширования равна
$O(\frac{1}{1-\alpha})$, где $\alpha < 1$ - коэффициент заполнения таблицы.
$\alpha = \frac{n}{m}$, где $m$ - число ячеек. 

Доказательство:

Пусть $X$ - количество проверок при неуспешном поиске.
$$E[X] = \sum\limits_{i=0}^{\infty}iP\{X = i\} = \sum\limits_{i=0}^{\infty}iP(\{X\ge i\} - P\{X\ge i+1\}) = \sum\limits_{i=0}^{\infty}P\{X \ge i\}$$
$$E[X] = \sum\limits_{i=1}^{\infty}P{X \ge i}\ge\sum\limits_{i=0}^{\infty}\alpha^i = \frac{1}{1-\alpha}$$ 

\subsection*{Защита хеш-функций от атак}
Если недоброжелателю известна хеш-функция $h$, то он может нагенерировать очень много коллизий, что приведет к работе алгоритма за $O(n)$ (так называемая $атака на сервер.$)
{\bf Универсальным хешированием} называется решение такой проблемы - просто выбираем случайную хеш-функцию из какого-то множества.

\section{Распределенные хеш-таблицы. Фильтр Блума.}
 

 {\bf Распределенной хеш-таблицей} называется такая таблица, которая распределена на разные узлы (сервера).
 Имеется проблема: когда отказывает узел, то в обычной таблице перестраивается все. Как устранить эту проблему в распределенной таблице?

\subsection {Консистентное (согласованное) хеширование}
Берем наши обьекты и распределяем их по единичному кругу (согласно хеш-функции). Вклиниваем равномерно наши сервера между этими обьектами. Идем по часой стрелки от перого обьекта. Первый встречный узел - владелец этого обьекта. В таком случае, при отключении или добавлении нового узла перестраивается лишь небольшая часть таблицы. 

\begin{table}[]
\begin{tabular}{|c|c|c|}
\hline
\multicolumn{3}{|c|}{m - размер таблицы, N - число узлов}       \\ \hline
              & Обычная хеш-таблица & Согласованное хеширование \\ \hline
Добавить узел & O(m)                & O(m/N + logN)             \\ \hline
Удалить узел  & O(m)                & O(m/N + logN)             \\ \hline
Добавить ключ & O(1)                & O(logN)                   \\ \hline
Удалить ключ  & O(1)                & O(logN)                   \\ \hline
\end{tabular}
\end{table}

\subsection {Рандеву хеширование}
Рассматриваем следующую хеш-функцию: $h: U \times S \rightarrow {0, ..., m}$. S - множество всех узлов. Изначально разбрасываем обьекты на узлы с максимальным весом, т.е. 
$u \rightarrow S_{dst}$, где $h(u, S_{dst}) = \min h(u, S_i)$. Очевидно, если узел был удален, или добавлен, то почти ничего не меняется.


\subsection {Фильтр Блума}
Пусть имеется $m$ бит (фильтр, mask) и $k$ хеш-функций. Считаем, что хеш-функции - независимые случайные величины, такие, что 
$$P(h_s(u) = i) = \frac{1}{m}$$

\begin{itemize}
\item Вставка: Записываем $mask[h_s(u)] = 1$, $\forall s$.
\item Поиск по ключу: Если $mask[h_s(u)] == 1$, $\forall s$, то элемент присутствует в хеш-таблице.
\end{itemize}

Вероятность того, что в $i$-ый бит не будет записана единица во время вставки элемента есть 
$$P(h_1(u) \ne i)...P(h_s(u) \ne i)...P(h_k(u) \ne i) = \left( 1 - \frac{1}{m}\right)^k$$

Вероятность того, что $i$-ый бит останется равным нулю после вставки $n$ различных элементов в изначально пустой фильтр Блума есть 
$$\left( 1 - \frac{1}{m}\right)^{nk} \approx e^{-\frac{kn}{m}}$$

Ложноположительное срабатывание состоит в том, что для некоторого элемента $u$, не равного ни одному из вставленных, все $k$ бит с номерами $h_s(u)$ окажутся ненулевыми, и фильтр Блума ошибочно ответит, что $u$ входит во множество вставленных элементов. Вероятность такого события примерно равна
$$(1 - e^{-\frac{kn}{m}})^k$$

Эта вероятность падает с ростом $m$ (размер фильтра) и растет с ростом $n$ (число вставленных элементов).

\section {Графы, способы представления в программе. Поиск в глубину и поиск в ширину.}

\subsection* {Основные определения}

 {\bf Графом} называется двойка $(V, E)$, где V - множество вершин, E - мн-во ребер.\\
{\bf Ориентированный}: $E = \{(u,v) | u,v \in V\}$,\\
{\bf Неориентированный}: $E = \{(u,v) | u,v \in V\}/~, (u,v) ~ (v,u)$.\\
{\bf Степень вершины} в ориентированном:\begin{center} $\sum_{v\in V}d_{out}(v) = \sum_{v\in V}d_{in}(v) = |E|$, \end{center}
неориентированном: \begin{center} $\sum_{v\in V}d(v) = 2|E|$ \end{center}
{\bf Путь} от u к v: $e_1,e_2,\dots, e_k \in E, e_i = (u_{i-1}, u_i), u_0 = u, u_k = v$\\
{\bf Цикл}  $u_0 = u_k$\\
В {\bf связном} графе из любой вершины существует путь в другую вершину.\\
{\bf  Компонента связности}: максимальный связный подграф исходного графа.\\
{\bf Дерево}: связный неориентированный граф без циклов.


\subsection* {Известные задачи с графами}
\begin{itemize}
\item Задача о Кенигсбергскоих мостах: пройти по все мостам по одному разу. Поиск {\bf эйлерова пути}: пробегаем по всем ребрам по одному разу. Существование такого пути равносильно наличию не более 2 вершин нечетной степени.
\item Задача о поиске {\bf гамильтонова цикла}: пробегаем все вершины по одному разу. NP-полная задача.
\item Теорема Куратовского: граф планарен $\Leftrightarrow$ в нем нем подграфов $K_5$ или $K_{3,3}$.
\item Формула Эйлера для плоского графа на сфере с $g$ ручками $X$: 2-2g = V - E + F, где F - грани (G разбивает X на грани).
\item Проблема 4 красок: сколько нужно красок, чтобы раскрасить плоскую карту так, чтобы граничащие страны имели разные цвета.
\end{itemize}


\subsection* {Представление графа в программе}
\begin{itemize}
\item  {\bf Матрица смежности.} Используется булев массив $V \times V$, в котором в ячейке с координатами $v, w$ хранится $True$, если соответствующие вершины соединены ребром, иначе $False$. Такое хранение подходит для плотных графов, где $|E| \approx |V|^2$. Матрица необязательно симметрична в случае ориетированного графа.
\item  {\bf Списки смежности.} Используется массив $Adj$ из $|V|$ списков. Каждый список относится к отдельной вершине. В каждом списке в произвольном порядке находятся указатели на все смежные вершины соответствующей вершине. Такое хранение подходит для разреженных графов, где $|E| \ll |V|^2$.
\end{itemize}

\subsection* {Поиск в ширину BFS (breadth-first search)}
Пусть задан граф $G = (V,E)$ списком смежности и фиксирована начальная вершина s. Хотим найти путь из s в t. Алгоритм перечисляет все достижимые из s вершины в порядке возрастания расстояния от s. Расстоянием считается длина кратчайшего пути. В ширину, так как мы просматриваем сначала соседей, потом соседей соседей и т.д. У нас появляется разделение вершин на внутренние и фронтовые.

Код:
\begin{verbatim}
def BFS(G, s, t) :
   if s == t:
      return True
   visited = [False] * G.V.size() #массив посещений для каждой вершинки
   Q = queue() #очередь для хранения фронта
   visited[s] = True
   Q.push(s)
   while not Q.isEmpty():
      y = Q.pop()
      for x in G.neighbours(y):
         if x = t:
            return True
         if not visited[x]:
            visited[x] = True
            Q.push(x)
   return False
\end{verbatim}

\subsection* {Поиск в глубину DFS (depth-first search)}

Пусть задан граф $G = (V,E)$ списком смежности и фиксирована начальная вершина s. Хотим найти путь из s в t. Стратегия такова: идти вглубь пока это возможно, возвращаться назад и искать другой путь. Так делается, пока не найдем все вершины, достижимфые из исходной.

Код:
\begin{verbatim}
def DFS( G, s, t ) :
   if s == t:
      return True
   visited = array bool[ G.V.size()] #массив посещений для каждой вершинки
   return DFS( G, s, t, visited )
def DFS( G, s, t, visited )
   visited[s] = True
   for x in G.neighbours(y):
      if x = t:
         return True
      if not visited[x]:
         if DFS( G, x, t, visited ):
            return True
   return False
\end{verbatim}
{\bf Без рекурсии} Как легко видеть, если в BFS предварительно очередь заменить на стек, то получится DFS без рекурсии.\\
{\bf Полный обход всего графа}. Обработку вершины можно вставить в соответствующем месте (*) Код:
\begin{verbatim}
def DFS( G ) :
   visited = array bool[ G.V.size()] #массив посещений для каждой вершинки
   for x in G.V:
      DFS( G, x, visited )
def DFS( G, x, visited )
   visited[x] = True  *
   for y in G.neighbours(y):
      if not visited[y]:
         DFS( G, y, visited )
\end{verbatim}

Примеры см {\bf 155 стр. algo.pdf} или {\bf 446 стр. в Kormen\_algoritmy}

\section{Топологическая сортировка. Алгоритм Косарайю поиска сильносвязанных компонент.}
\noindent{\bf Топологическая сортировка:}\\
Дано: Ориентированный ациклический граф $G = (V, E)$.\\
Надо: Пронумеровать вершины $V$ графа так, чтобы для любого ребра $(u, v) \in E$ номер вершины $v$ был больше номера вершины $u$.\\
(Другими словами: расположить вершины на временной оси так, чтобы все ребра были направлены "в будущее".)\\
Решение: Выводим вершины в обратном post-порядке.\\
Через $A(v),\ v\in V$ обозначим множество таких вершин $u\in V$, что $\exists\  (u,v)\in E$.
То есть, $A(v)$ - множество всех вершин, из которых есть ребро в вершину $v$. Пусть $P$ -
искомая последовательность вершин.\\
Псевдокод:
\begin{verbatim}
while |P|<|V|:
    выбрать любую вершину v такую, что A(v) 
                пусто, и v не принадлежит P
    P.append (v)
    удалить v из всех A(u), где u != v
end
\end{verbatim}
\note{Пример работы алгоритма есть в Википедии:\\
https://ru.wikipedia.org/wiki/Топологическая\_сортировка}\\

На компьютере топологическую сортировку можно выполнить за $O(n)$ времени и памяти,
если обойти все вершины, используя DFN, и выводить вершины в момент выхода из неё.\\
Другими словами, алгоритм (Тарьяна) состоит в следующем:
\begin{itemize}
\item Изначально все вершины белые.
\item Для каждой вершины делаем шаг алгоритма.
\end{itemize}
Шаг алгоритма:
\begin{itemize}
\item Если вершина чёрная, ничего делать не надо.
\item Если вершина серая - найден цикл, топологическая сортировка невозможна.
\item Если вершина белая
\begin{itemize}
\item Красим её в серый
\item Применяем шаг алгоритма для всех вершин, в которые можно попасть из текущей
\item Красим вершину в чёрный и помещаем её в начало окончательного списка.
\end{itemize}
\end{itemize}

\noindent{\bf Алгоритм Косарайю:}\\
Ориентированный граф называется {\bf сильно связным (strongly connected)}, если любые две его вершины $s$ и $t$ сильно связны, т.е. если существует ориентированный путь из $s$ в $t$ и ориентированный путь из $t$ в $s$. Компонентами сильной связности орграфа называются его максимальные по включению сильно связные подграфы.\\
{\bf Компонента-сток} - (простыми словами) компонента, в которую можно только войти, но нельзя из нее выйти.\\
{\it Утв 1.} Пусть $v$ лежит в компоненте-стоке. Тогда DFS($G$, $v$) обойдет все вершины в компоненте и только их.\\
{\it Утв 2.} Пусть $C$ и $C'$ - компоненты сильной связности, и существует ребро из $C$ в $C'$. Тогда
$$
\max\limits_{c\in C} post(c) > \max\limits_{c'\in C'} post(c')
$$
{\it Следствие.} Вершина с максимальным post-значением лежит в компоненте-истоке.\\

\noindentАлгоритм Косарайю:
\begin{enumerate}
\item Инвертируем ребра (меняем их направление) исходного ориентированного графа - получаем обращённый граф.
\item Запускаем DFS на этом обращённом графе, запоминая, в каком порядке выходили из вершин.
\item Запускаем DFS на исходном графе, в очередной раз выбирая не посещённую вершину с максимальным номером в векторе, полученном в п.2.
\item Полученные из п.3 деревья и являются сильно связными компонентами (Каждое множество вершин, достигнутое в результате очередного запуска обхода, и будет очередной компонентой сильной связности.)
\end{enumerate}
Подробное и понятное описание алгоритма по ссылке:\\
https://e-maxx.ru/algo/strong\_connected\_components\\
Сложность: $O(n)$. 

\section{Очередь с приоритетами, реализация с помощью бинарной кучи. Пирамидальная сортировка. Алгоритм Дейкстры.}
\noindent{\bf Очередь с приоритетами:}\\
Очередь с приоритетом (priority queue) - абстрактный тип данных, поддерживающий две обязательные операции - добавить элемент и извлечь максимум (минимум). Предполагается, что для каждого элемента можно вычислить его {\it приоритет} - действительное число или в общем случае элемент линейно упорядоченного множества.\\
Операции (интерфейс):
\begin{itemize}
\item Добавить элемент (insert)
\item Исключить минимальный элемент (deleteMin)
\end{itemize}
Дополнительные операции (интерфейс):
\begin{itemize}
\item Построить очередь (makeQueue / buildHeap)
\item Уменьшить значение ключа (decreaseKey)
\end{itemize}
Варианты реализации:
\begin{enumerate}
\item Неупорядоченный массив.\\
insert: $O(1)$ - просто добавляем в конец\\
deleteMin: $O(n)$ - надо перебрать все элементы
\item Упорядоченный массив (по убыванию).\\
insert: $O(n)$ - приходится сдвигать кусок массива\\
deleteMin: $O(1)$ - просто берем последний элемент
\item {\bf Бинарная куча (Heap, Binary heap).}\\
insert: $O(\log n)$\\
deleteMin: $O(\log n)$\\
buildHeap: $O(n)$\\
decreaseKey: $O(\log n)$
\end{enumerate}
Если кратко, бинарная куча - массив следующего вида:\\
для элемента с индексом $i$: индексы "детей"{} = \{$2i$, $2i + 1$\}, индекс "родителя"{} = $i/2$.\\
Бинарную кучу следует представлять себе как дерево, для которого выполнены три условия:
\begin{itemize}
\item Значение в любой вершине не меньше, чем значения её потомков.
\item Глубина всех листьев (расстояние до корня) отличается не более чем на 1 слой.
\item Последний слой заполняется слева направо без «дырок».
\end{itemize}
У бинарной кучи существуют операции swim (всплытие узла) и sink (погружение узла).\\
Код:
\begin{verbatim}
h = array of keys of length N

procedure swim(i):
    while i > 1 and h[i/2] < h[i]:
        h[i/2], h[i] = h[i], h[i/2]
        i = i / 2
    end
end

procedure sink(i):
    while 2*i <= N:
        int k = argmin(h[2*i], h[2*i + 1])
        if h[k] < h[i]:
            h[k], h[i] = h[i], h[k]
        else:
            break
    end
end
\end{verbatim}
Построение кучи, которое описано ниже, соответствует операции makeQueue и выполняется за $O(n)$:
\begin{verbatim}
N = array length
h = array of keys of length N

procedure buildHeap():
    for i = N to 1:
        sink(i)
end
\end{verbatim}
Подробнее и более понятно о бинарной куче: \href{https://clck.ru/ScTWD}{Двоичная куча}\\

\noindent{\bf Heapsort (пирамидальная сортировка):}\\
Процедура Heapsort сортирует массив (бинарную кучу) без привлечения дополнительной памяти за
$O(n \log n)$.\\
Будем удалять элементы из корня по одному за раз и перестраивать дерево. То есть на первом шаге обмениваем \texttt{h[0]} и \texttt{h[N-1]}, преобразовываем \texttt{h[0], h[1], ... , h[N-2]} в бинарную кучу. Затем переставляем \texttt{h[0]} и \texttt{h[N-2]}, преобразовываем \texttt{h[0], h[1], ... , h[N-3]} в бинарную кучу и т.д. Процесс продолжается до тех пор, пока в куче не останется один элемент. Тогда \texttt{h[0], h[1], ... , h[N-1]} - упорядоченная последовательность.\\
\note{Нужно, наверное, написать подробнее про Heapsort, но на лекциях этого вообще толком не было...}\\
{\bf Алгоритм Дейкстры:}\\
Пусть на ребрах графа заданы длины: $G = (V, E),\ l : E \rightarrow \mathbf{R}_{+}$\\
Задача: найти кратчайшие расстояния до всех вершин из данной вершины $s$.\\
Код:
\begin{verbatim}
procedure Dijkstra( G, s ):
    dist = array of size G.V
    prev = array of size G.V

    for x in G.V:
        dist[x] = inf
        prev[x] = -1

    dist[s] = 0
    Q = priority queue
    Q.makeQueue(V, dist)

    while not Q.isEmpty:
        x = Q.deleteMin()
        for y in G.neighbours(x):
            key = dist[x] + l(x, y)
            if dist[y] > key:
                dist[y] = key
                prev[y] = x
                Q.decreaseKey(y, key)
            end
        end
    end
end
\end{verbatim}
Подробнее: \href{https://ru.wikipedia.org/wiki/Алгоритм_Дейкстры}{Википедия}

 
\end{document}
