%main.tex
\documentclass[a4paper,12pt]{article}
\usepackage[utf8]{inputenc}
\usepackage[english,russian]{babel}
\usepackage{a4wide}
\usepackage{mathtext}
\usepackage{amsbsy}
\usepackage{amsthm}
\usepackage{amsmath}
\usepackage{amssymb}
\usepackage{amsfonts}
\usepackage{tikz}
\usepackage{verbatim}
\usepackage{graphicx}
 
\begin{document}

\tableofcontents
\section {Программа экзамена}
\begin{enumerate}
\item Сложность алгоритмов, бинарный поиск, сортировка выбором и вставками
\item Сортировка слиянием. Основная теорема для метода «разделяй и властвуй».
\item Алгоритм Карацубы умножения чисел, алгоритм Штрассена умножения матриц. Быстрое преобразование Фурье.
\item Оценка снизу количества сравнений при сортировке, быстрая сортировка Хоара, сложность в среднем и в худшем. Задача Дейкстры о голландском флаге, 3х-частное разбиение, эвристики выбора опорного элемента.
\item Алгоритмы нахождения k-й порядковой статистики - вероятностный и детерминированный. Метод сортировки TimSort.
\item Абстрактные типы данных, интерфейс и реализация, стек и очередь - реализация связанным списком и массивом.
\item Ассоциативные массивы, бинарные деревья поиска.
\item Сбалансированные деревья, 2-3 и красно-черные деревья.
\item B-деревья.
\item Хеш-таблицы, реализация методом цепочек и открытой адресацией.
\item Распределенные хеш-таблицы. Фильтр Блума.
\item Графы, способы представления в программе. Поиск в глубину и поиск в ширину.
\item Топологическая сортировка. Алгоритм Косарайю поиска сильносвязанных компонент.
\end{enumerate}
 
\end{document}
